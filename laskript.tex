\documentclass[a4paper, titlepage]{article}
\usepackage[utf8]{inputenc}
\usepackage[T1]{fontenc}
\usepackage{textcomp}
\usepackage[german]{babel}
\usepackage{amsmath, amssymb, amsthm}
\usepackage{mdframed}
\usepackage{fancyhdr}
\usepackage{geometry}
\usepackage{import}
\usepackage{pdfpages}
\usepackage{transparent}
\usepackage{xcolor}
\usepackage{array}
\usepackage{enumerate}
\usepackage{gauss}
\usepackage{tabulary}
\usepackage{subcaption}
\usepackage{tikz}
\usepackage{pgfplots}
\usepackage[nobottomtitles]{titlesec}
\usepackage{listings}
\usepackage{mathtools}
\usepackage[hidelinks]{hyperref}
\usepackage{blindtext}
\usepackage{xpatch}



\usetikzlibrary{quotes, angles}

\DeclareOption*{\PassOptionsToClass{\CurrentOption}{article}}
\DeclareOption{uebung}{
	\makeatletter
	\lhead{\@title}
	\rhead{\@author}
	\makeatother
}
\ProcessOptions\relax

% PAGE GEOMETRY
\geometry{
	left=30mm,
	right=30mm,
	top=25mm,
	bottom=20mm
}

% PARAGRAPH no indent but skip
\setlength{\parskip}{3mm}
\setlength{\parindent}{0mm}

\theoremstyle{definition}
\newmdtheoremenv{satz}{Satz}[section]
\newmdtheoremenv{lemma}[satz]{Lemma}
\newmdtheoremenv{korrolar}[satz]{Korrolar}
\newmdtheoremenv{definition}[satz]{Definition}

\newtheorem{Folgerung}[satz]{Folgerung}
\newtheorem{bsp}[satz]{Beispiel}
\newtheorem{bem}[satz]{Bemerkung}
\newtheorem{aufgabe}{Aufgabe}
\newtheorem{alg}[satz]{Algorithmus}
\newtheorem*{anm}{Anmerkung}
\newtheorem*{ziel}{Ziel}
\newtheorem*{fg}{Folgerung}
\newtheorem*{frage}{Frage}
\newtheorem*{erin}{Erinnerung}
\newtheorem*{notation}{Notation}

\newcommand{\N}{\mathbb{N}}
\newcommand{\R}{\mathbb{R}}
\newcommand{\Z}{\mathbb{Z}}
\newcommand{\Q}{\mathbb{Q}}
\newcommand{\C}{\mathbb{C}}
\newcommand{\Fit}{\mathrm{Fit}}
\newcommand{\GL}{\mathrm{GL}}
\newcommand{\contr}{\text{\Large\lightning}}
\newcommand{\End}{\mathrm{End}}
\newcommand{\M}{\mathrm{M}}
\newcommand{\ggT}{\mathrm{ggT}}


% HEADERS

\pagestyle{fancy}

\newcommand{\incfig}[1]{%
	\def\svgwidth{\columnwidth}
	\import{./figures/}{#1.pdf_tex}
}
\pdfsuppresswarningpagegroup=1

% horizontal rule
\newcommand\hr{
	\noindent\rule[0.5ex]{\linewidth}{0.5pt}
}

% punkte tabelle
\newcommand\punkte{
	\begin{tabular}{|c|m{1cm}|m{1cm}|m{1cm}|m{1cm}|m{1cm}|@{}m{0cm}@{}}
		\hline
		Aufgabe & \centering A1 & \centering A2 & \centering A3 & \centering A4 & \centering $\sum$ & \\[5mm] \hline
		Punkte & & & & & & \\[5mm] \hline
	\end{tabular}
}

% code listings, define style
\lstdefinestyle{mystyle}{
	commentstyle=\color{gray},
	keywordstyle=\color{blue},
	numberstyle=\tiny\color{gray},
	stringstyle=\color{black},
	basicstyle=\ttfamily\footnotesize,
	breakatwhitespace=false,         
	breaklines=true,                 
	captionpos=b,                    
	keepspaces=true,                 
	numbers=left,                    
	numbersep=5pt,                  
	showspaces=false,                
	showstringspaces=false,
	showtabs=false,                  
	tabsize=2
}

% activate my colour style
\lstset{style=mystyle}

% better stackrel
\let\oldstackrel\stackrel
\renewcommand{\stackrel}[2]{%
	\oldstackrel{\mathclap{#1}}{#2}
}%

% integral d sign
\makeatletter \renewcommand\d[1]{\ensuremath{%
		\;\mathrm{d}#1\@ifnextchar\d{\!}{}}}
\makeatother
\title{Lineare Algebra II Skript}
\author{Prof. Vogel}
\setcounter{section}{-1}


\begin{document}
\maketitle
\tableofcontents
\newpage
\part{Unitäre Räume}
	Ziel: Entwicklung einer analogen Theorie zur reellen Theorie der euklidischen VR für $\C$-VR
\section{Unitäre Räume und der Spektralsatz}
	Notation: In diesem Abschnitt sei $V$ stets ein endlicher $\C$-VR.
\begin{definition}
	$h: V \times V \longrightarrow $ heißt eine \textbf{Sesquilinerform} auf $V$\\ $\overset{\text{Def}}{:\Leftrightarrow}$
	\begin{enumerate}[(S1)]
		\item h ist linear im ersten Argument, d.h.
	\begin{itemize}
			\item $h(v_1 + v_2, w) = h(v_1, w) + h(v_2,w)$,
			\item $h(\lambda v, w) = \lambda h(v,w), $
	\end{itemize}
	$\forall v_1, v_2, w \in V, \lambda \in \C.$
	\item h ist semilinear im zweiten Argument, d.h.
		\begin{itemize}
			\item $h(v, w_1 +w_2) = h(v, w_1) + h(v, w_2)$
			\item $h(v, \lambda w) = \overline{\lambda} h(v,w) $
		\end{itemize}
	$\forall v, w_1 ,w_2 \in V, \lambda \in \C.$		
\end{enumerate} 
\end{definition}
\begin{anm}
sesqui = 1,5.
In der Literatur sind (S1) und (S2) gelegentlich vertauscht.
\end{anm}
\begin{bsp}
	$\C, h(x,y) = x^t\overline{y} $ ist eine Sesquilinearform auf $\C^n:$
	\begin{align*}
	(x_1 + x_2)^ty &= x_{1}^ty+x_{2}^ty,\\
	(\lambda x)^{t}y &= \lambda(x^ty), \\
	x^t(\overline{y_1+y_2}) &= x^t\overline{y_1}+x^t\overline{y_2},\\
	 x^t \overline{\lambda y}&= \overline{\lambda} x^ty.\\
		\text{für } x_1, x_2,& y , y_1, y_2 \in \C^n.
	\end{align*}
	$h$ ist für $n>0$ keine Bilinearform: $$h(\begin{pmatrix} 1 \\ 0 \\ \vdots \\ 0 \end{pmatrix}
	, i\begin{pmatrix} 1 \\ 0 \\ \vdots \\ 0 \end{pmatrix}) = (1, \cdots, 0) \begin{pmatrix} -i \\ 0 \\ \vdots \\ 0 \end{pmatrix} = -i \neq ih(\begin{pmatrix} 1 \\ 0 \\ \vdots \\ 0 \end{pmatrix},\begin{pmatrix} 1 \\ 0 \\ \vdots \\ 0 \end{pmatrix}) = i.$$
\end{bsp}
\begin{definition}
    Sei $V $ ein $\C$-VR, $h$ Sesquilinearform auf $V$. $h$ heißt \textbf{hermitisch} $ \overset{\text{Def:}}{\Leftrightarrow} h(w,v) = \overline{h(v,w)}$ für alle $v,w \in V.$\\
\end{definition}
\begin{anm}
    In diesem Fall ist $h(v,v) = \overline{h(v,v)}$ für alle $V \in V$, d.h. $h(v,v) \in \R$ für alle $v\in V.$
\end{anm}
\begin{bsp}
	$h(x,y) = x^t\overline{y} $ aus Bsp. 0.2 ist hermitesch, denn es ist $h(y,x) =\underbrace{ y^t\overline{x}}_{\in \C}=(y^t\overline{x})^t = \overline{x}^t(y^t)^t = \overline{x}^ty= x^t \overline{y} = \overline{h(x,y)}.$\\
	Hier ist $h(x,x) = x^t\overline{x} = (x_1,...,x_n)\begin{pmatrix} \overline{x_1} \\  \vdots \\ \overline{x_n} \end{pmatrix} = x_1\overline{x_1} + ... +x_n \overline{x_n} = |x_1|^2 + ... + |x_n|^2 \in \R.$
\end{bsp}
\begin{definition}
	Sei $h: V\times V \longrightarrow \C $ eine Sesquilinearform, $B = (v_1,...,v_n)$ Basis von $V$.
	$$M_B = (h(v_i,v_j))_{1 \leq i,j \leq n} \in M_{n,n}(\C)$$ heißt die \textbf{Fundamentalmatrix} von $h$ bzgl. $B$. (Darstellungsmatrix)
\end{definition}
\begin{bsp}
	Für $h(x,y) = x^t\overline{y}$ aus Bsp. 0.2, ist $$M_B(h)=\left(\begin{matrix}
	{1}&{}&{0 }\\
	{  }&{\ddots}&{  }\\
	{ 0 }&{}&{ 1 }
	\end{matrix}\right) = E_n
	 $$
\end{bsp}
\begin{definition}
Sei $M \in M_{n,n}(\C).$ $M^{*} := \overline{M}^t$ heißt die zu M \textbf{adjungierte Matrix}. M heißt \textbf{hermitesch} $\overset{\text{Def:}}{\Leftrightarrow}M = M^{*}$\\
\end{definition}
\begin{anm}
     Nicht verwechseln mit der adjunkten Matrix!
\end{anm}    
\begin{satz}
	Sei $B = (v_1,...,v_n)$ eine Basis von $V$.\\ $\operatorname{Sesq(V)} := \{ h: V \times V \longrightarrow \C  | h\text{ ist Sesquilinearform} \}$ ist ein $\C$-VR. (UVR von $\C$-VR $\operatorname{Abb}(V\times V, \C$). Die Abbildung $$M_B = \operatorname{Sesq}(V) \longrightarrow M_{n,n}(\C), h \mapsto M_B(h)$$ ist ein Isomorphismus von $\C$-VR mit Umkehrabbildung $$h_B: M_{n,n}(\C) \longrightarrow \operatorname{Sesq}(V), A \mapsto h_B(A) \text{ mit } h_B(A): V \times V \longrightarrow\C, $$ $$\left(\sum_{i=1}^{n}x_iv_i,
	 \sum_{j=1}^{n}y_jv_j\right)\mapsto x^tA\overline{y} \text{ mit }  x = \begin{pmatrix} x_1 \\ \vdots \\ x_n \end{pmatrix}, y = \begin{pmatrix} y_1 \\ \vdots \\ y_n \end{pmatrix}$$
	 Es gilt: $h$ hermitesch $\Leftrightarrow M_B(h)$ hermitesch.
	 \end{satz}
	 \begin{proof}\ 
	 	\begin{itemize}
	 		\item $h_B$ ist wohldefiniert: $h_B(A)$ ist Sesquilinearform analog zur Rechnung in Bsp. 0.2.
	 		\item $M_B, h_B$ sind $\C$-linear: klar.
	 		\item $M_B \circ h_B = id$, denn: Sei $A = (a_{ij}) \in M_{n,n}(\C) \Rightarrow h_B(A)(v_i,v_j) = e_i^tA\overline{e_j} = a_{ij}, $ d.h. Darstellungsmatrix von $h_B(A)$ bzgl. $B$ ist $A$.
	 		\item $h_B \circ M_B = id$, denn: Sei $h\in \operatorname{Sesq}(V) \Longrightarrow h_B(M_B(h))(v_i,v_j) = e_{i}^tM_B(h)\overline{e_j} = h(v_i, v_j) \Longrightarrow h_b(M_B(h)) = h.$
	 		Für $h\in \operatorname{Sesq}(V) $ ist 
	 		\begin{align*}
	 		\text{h hermitesch} \Leftrightarrow h(w,v) &= \overline{h(v,w)}\text{ für alle } v,w \in V\\
	 		& \Leftrightarrow h(v_j,v_i) = \overline{h(v_i,v_j)} \text{ für alle } i=1,...n \\
	 		& \Leftrightarrow M_B(h)^t = \overline{M_B(h)}\\
	 		& \Leftrightarrow M_B(h) = \overline{M_B(h)}^t = M_B(h)^{*}
	 		\end{align*}
	 	\end{itemize}
	 \end{proof}
\begin{satz}
	$A,B$ Basen von V, $h$ Sesquilinearform auf $V$. Dann gilt 
	$$M_B(h) = (T_{A}^{B})^tM_B(h)\overline{T_{A}^{B}}, \text{ wobei } T_{A}^{B} = M_{A}^{B}(id_V).$$
\end{satz}
\begin{proof}
	analog zum reellen Fall
\end{proof}
\begin{definition}
	Sei $h$ hermitesche Form. $h$ heißt positiv definit $\overset{\text{Def:}}{\Leftrightarrow} h(v,v)>0, \forall v \in V , v>0.$
	Eine positiv definite hermitesche Sesquilinearform nennt man auch ein komplexes \textbf{Skalarprodukt}.
\end{definition}
\begin{bsp}
	$V = \C^n, \langle \cdot, \cdot \rangle :\C^n \times C^n, \langle x,y \rangle := x^t\overline{y}$ ist ein Skalarprodukt (Standartskalarprodukt auf $\C^n$)
	denn: $$\langle x,x \rangle = |x_1|^2 + ... + |x_n|^2 > 0 ,\forall x = \begin{pmatrix} x_1 \\ \vdots \\ x_n \end{pmatrix} \in \C, x \neq 0.$$
\end{bsp}
\begin{definition}
	Ein unitärer Raum ist ein Paar $(V,h)$ bestehend aus einem endlichdimensionalen $\C$-VR $V$ und einem Skalarprodukt $h$ auf $V$.
\end{definition}
\begin{definition}
	Sei $(V,h)$ unitärer Raum, $v\in V.$
	$$||v|| := \sqrt{\langle v,v \rangle} \text{ heißt die Norm von } V.$$
\end{definition}
\begin{satz}
	Sei $(V,h)$ ein unitärer Raum. Dann gilt:
	\begin{enumerate}
		\item $||x+y|| \leq ||x||+||y||, \forall x,y \in V \text{ (Dreiecksungleichung)}$
		\item $|h(x,y)| \leq ||x|| \cdot ||y||, \forall x,y \in V \text{ (Cauchy-Schwarz-Ungleichung) }$
	\end{enumerate}
\end{satz}
\begin{proof}\ \\
		2. Seien $x,y \in V$. Falls $x =0$, dann $$h(x,y) = h(0,y) = h(0\cdot 0, y) = 0\cdot h(0,y) = 0 = || 0|| \cdot ||y||.$$
		Im Folgenden sei $x \neq 0$.
		 Setze 
		 \begin{align*}
			\alpha := \frac{h(x,y)}{||x||^2}, w := y-\alpha x 
			&\Rightarrow h(w,x) = h(y-\alpha x , x) = h(y- \frac{h(y,x)}{||x||^2}x,x) \\ 
			&= h(y,x)-\frac{h(y,x)}{||x||^2}\underbrace{h(x,x)}_{||x||^2}=0\\
			&\Rightarrow ||y||^2 = ||w +\alpha x||^2 = h(w+\alpha x, w + \alpha x) = ||w||^2 + \alpha \cdot \overline{\alpha}h(x,x)\\ 
			& = ||w||^2 + |a|^2||x||^2\\
			& \Rightarrow ||y|| \geq |a|||x|| = \frac{|h(y,x)|}{||x||^2}||x|| = \frac{|h(x,y)|}{||x||}\\
			&\Rightarrow ||y||||x|| \geq |h(x,y)|
		\end{align*}
		1. \begin{align*}
		||x+y||^2 = h(x+y,x+y) &= ||x||^2+||y||^2+h(x,y)+h(y,x)\\
		&=||x||^2+||y||^2+2\operatorname{Re}h(x,y)\\
		&\leq ||x||^2+||y||^2+2|h(x,y)|\\
		&\leq ||x||^2+||y||^2+2||x||||y||\\
		&= (||x||+||y||)^2
	\end{align*}
\end{proof}
\newpage
\begin{definition}
	Sei $(v_1,..,v_n)$ eine Basis von $V$. $(v_1, \dots,v_n) $ heißt eine
	\begin{align*} &\text{\textbf{Orthogonalbasis} von } V \overset{\text{Def:}}{\Leftrightarrow} h(v_i,v_j) = 0 \text{ für } i \neq j.\\
		&\text{\textbf{Orthonormalbasis} von V}\overset{\text{Def:}}{\Leftrightarrow} h(v_i, v_j) = \delta_{ij} \text{ für alle } 1 \leq i,j \leq n.
	\end{align*}
\end{definition}
\begin{satz}
	Sei $(V,h)$ ein unitärer Raum. Dann hat $V$ eine ONB.
\end{satz}
\begin{proof}
	\begin{itemize}
	gzz.: $(V,h)$ hat eine OB (normieren der Basisvektoren liefert dann ONB)
	Beweis per Induktion nach $n = \operatorname{dim}(V)$.
	\item $n = 0,1: $ trivial
	\item	$n\geq 2$: Wähle $v_1 \in V, v_1 \neq 0$
	Setze $H:=\{ w \in V | h(w,v_1) =0\}.$ 
	\begin{align*}
	&\text{Die Abbildung } \phi: V \longrightarrow \C, w \mapsto h(w,v_1) \text{ ist Linearform mit } \operatorname{ker}\phi = H\\
	&\Rightarrow\operatorname{dim} H = \operatorname{dim} \operatorname{ker}\phi = \operatorname{dim} V - \underset{\in\{0,1\}}{\operatorname{dim}\operatorname{im}\phi} \in \{n,n-1\}.\\
	&\text{Wegen } h(v_1,v_1)>0 \text{ ist } v_1 \not\in H; \text{ somit } \operatorname{dim} H = n-1 \\
	& (H,h\mid_{H \times H}) \text{ ein unitärer Raum der Dimension } n-1\\
	&\Rightarrow H \text{ hat OB } (v_2,..,v_n)\\
	&\Rightarrow (v_1, v_2,...,v_n) \text{ ist OB von V}
	\end{align*}
	\end{itemize}
\end{proof}
\begin{anm}
 Gram-Schmidt-Verfahren (wie über $\R$) liefert Algorithmus zur Bestimmung einer ONB.
\end{anm}
 \begin{definition}
	Sei $(V,h)$ ein unitärer Raum, $U \subset V $ ein Untervektorraum.
	$U^{\bot} = \{ v \in V | h(v,u) = 0 \text{ für alle } u \in U\}$ heißt das \textbf{orthogonale Komplement} zu $U$.
	$U,W$ sind Untervektorräume von $V$ mit $V = U \oplus W$ und $h(u,w) = 0 $ für alle $ u \in U, w \in W.$
	Dann heißt $V$ die \textbf{orthogonale direkte Summe } von $U$ und $W.$
	Notation: $V=U \hat{\oplus} W.$
\end{definition}
\begin{satz}
	Sei $(V,h)$ ein unitärer Raum, $U \subset V $ ein Untervektorraum. Dann gilt:
	$$V = U \hat{\oplus} U^{\bot}.$$
\end{satz}
\begin{proof}
	\begin{enumerate}
		\item \begin{align*}\text{Beh.: } &V = U + U^{\bot}\\
		&\text{Sei } (u_1,...,u_m)  \text{ ONB  von }U. \\
		&\text{Sei }v\in V.\text{ Setze } v' := v - \sum_{j=1}^{m}h(v,u_j)u_j\\
		&\text{Für } i= 1,...,m \text{ ist } h(v',u_i) = h(v,u_i) - \sum_{j=1}^{m}h(v,u_j)\underbrace{h(u_j,u_i)}_{=\delta_{ij}} = h(v,u_i)-h(v,u_i) = 0\\
		&\Rightarrow v' \in U^{\bot}\\
		&v = \underbrace{v'}_{\in U^{\bot}}+\underbrace{\sum_{j=1}^{m}h(v,u_j)u_j}_{\in U} \in U + U^{\bot}
	\end{align*}
		\item $U \cap U^{\bot} = 0,$ denn: $ u \in U \cap U^{\bot} \Rightarrow h(u,u) =0 \Rightarrow u=0.$
		\item Wegen 1. und 2. ist $V = U \hat{\oplus} U^{\bot}$, außerdem ist $h(u,u') = 0$ für $u\in U, u^{'} \in U^{\bot},$ somit $V = U \hat{\oplus} U^{\bot}$.
	\end{enumerate}
\end{proof}
\begin{definition}
	Seien $(V,h_v), (W,h_w)$ unitäre Räume, $\varphi: V \longrightarrow W$ eine lineare Abbildung. $\varphi $ heißt \textbf{unitär} $\Leftrightarrow h_w(\varphi (v_1),\varphi(v_2)) = h_v(v_1,v_2)$ für alle $v_1,v_2 \in V.$\\
	\textit{\textbf{Anmerkung}}: Ist $\varphi \in \operatorname{End}(V)$ ein unitärer Endomorphismus, dann ist $\varphi$ ein Isomorphismus, denn:
	\begin{itemize}
		\item $\varphi$ ist injektiv, wegen $\varphi(v)=0 \Rightarrow 0 = h(\varphi(v),\varphi(v)) = h(v,v) \Rightarrow v= 0$
		\item wegen $\operatorname{dim} V < \infty \text{ folgt } \varphi \text{ surjektiv. }$
	\end{itemize}
\end{definition}
\begin{bem}
	Sei $(V,h)$ unitärer Raum, $B=(v_1,...,v_n)$ ONB von $(V,h).$ Dann ist die Abbildung $$(\C^n,\langle\cdot,\cdot\rangle)\longrightarrow (V,h), e_i \mapsto v_i $$ ein unitärer Isomorphismus, d.h. $(V,h)$ ist unitär isomorph zu $(\C^n,\langle\cdot,\cdot\rangle).$
	\begin{proof}
		$h(\varphi(e_i),\varphi(e_j)) = h(v_i,v_j) = \delta_{ij} = \langle e_i,e_j\rangle$
	\end{proof}
\end{bem}
\begin{definition}
	Sei $A\in M_{n,n}(\C).$ 
	\begin{itemize}
		\item $A$ heißt
		\textbf{unitär}$\overset{\text{Def:}}{\Leftrightarrow} A^*A=E_n $
		\item $U(n) :=\{ A \in M_{n,n}(\C) | A \text{ ist unitär }\}$
		\item $U(n)$ ist eine Gruppe bzgl. "$\cdot$", die \textbf{unitäre Gruppe} vom Rang n.
		\item $\operatorname{SU}(n) := \{ A \in U(n) | \operatorname{det}(A) = 1\}$ ist eine Untergruppe von $U(n)$, die \textbf{spezielle unitäre Gruppe}.
	\end{itemize}
\end{definition}
\begin{bem}
	Sei $A\in M_{n,n}(\C)$. Dann sind äquivalent:
	\begin{enumerate}
		\item[(i)] $A$ ist unitär 
		\item[(ii)]Die Abbildung $(\C^n, \langle \cdot, \cdot \rangle) \longrightarrow ( \C^n, \langle \cdot, \cdot \rangle), x \mapsto Ax$ ist unitär. Hierbei ist $\langle \cdot, \cdot \rangle $ das Standart-skalarprodukt.
	\end{enumerate}
\begin{proof}
	$\langle Ax, Ay\rangle = (Ax)^t\overline{Ay}=x^tA^t\overline{A}\overline{y}$
	\begin{align*}
	&\text{Somit ist die Abbildung aus (ii) unitär }\\
	&\Leftrightarrow x^tA^t\overline{A}\overline{y} = \langle x,y \rangle = x^t\overline{y} \text{ für alle } x,y\in \C^n\\
	&\Leftrightarrow h_{(e_1,...,e_n)}(A^t,\overline{A}) = h_{(e_1,...,e_n)}(E_n) \text{ (vgl. Satz 0.7 )}\\
	& \overset{0.7}{=} A^tA = E_n \Leftrightarrow \overline{A}^t(A^t)^t = E_n \Leftrightarrow \overline{A}^tA = A^{*}A=E_n \Leftrightarrow A \text{ ist unitär}
	\end{align*}	
\end{proof}
\end{bem}
\begin{bem}
	Sei $(V,h)$ ein unitärer Raum und $f \in \operatorname{End}(V).$ Dann existiert genau ein $f^{*}\in \operatorname{End}(V)$ mit 
	$$h(f(x),y) = h(x,f^{*}(y)), \forall v,y \in V$$
	$f^{*}$ heißt die \textbf{zu f adjungierte Abbildung}. Ist $B$ eine ONB von $(V,h)$, dann ist $$M_B(f^{*})=M_B(f)^{*}$$
	\begin{proof}
		analog zu LA1, 19/20; Def. + Lemma 5.48
	\end{proof}
\end{bem}
\newpage
\begin{definition}
	Sei $(V,h)$ ein unitärer Raum, $f\in \operatorname{End}(V), A\in M_{n,n}(\C).$ 
	\begin{itemize}
		\item $f$ heißt \textbf{selbstadjungiert} $\overset{\text{Def:}}{\Leftrightarrow}f^{*} = f$
		\item $f$ heißt \textbf{normal} $\overset{\text{Def:}}{\Leftrightarrow} f^{*}\circ f = f \circ f^{*}$
		\item $A$ heißt \textbf{selbstadjungiert} $\overset{\text{Def:}}{\Leftrightarrow} A^{*}=A$
		\item $A$ heißt \textbf{normal} $\overset{\text{Def:}}{\Leftrightarrow} A^{*}A=AA^{*}$
	\end{itemize}
\begin{anm}
    $A$ ist selbstadjungiert $\Leftrightarrow A $ ist hermitisch.
\end{anm}
\end{definition}
\begin{bem}
	Sei $(V,h)$ ein unitärer Raum, $f \in \operatorname{End}(V)$.
	Dann gilt: \begin{enumerate}[(a)]
		\item $f$ unitär $\Rightarrow f $ normal
		\item $f$ selbstadjungiert $\Rightarrow f $ normal
	\end{enumerate}
Für $A\in M_{n,n}(\C)$ gilt: $A$ unitär $\Rightarrow A$ normal, $A$ selbstadjungiert $\Rightarrow A$ normal
\end{bem}
\begin{proof}
	\begin{enumerate}[(a)]
		\item Seien $v,w \in V$ 
		\begin{align*}
		\underset{f \text{ Isomorphismus, da unitär }}{\Rightarrow} &h(v,f^{-1}(w)) \underset{\text{ f unitär}}{=}h(f(v),f(f^{-1}(w)))=h(f(v),w)\\
		\underset{0.23}{\Rightarrow} &f^{*} = f^{-1} \Rightarrow f^{*}\circ f = f^{-1} \circ f = id_V = f \circ f^{-1} = f \circ f^{*}
		\end{align*}
		\item $f$ selbstadjungiert $\Rightarrow f^{*} = f \Rightarrow f^{*} \circ f = f \circ f = f \circ  f^{*}$
	\end{enumerate}
\end{proof}
\begin{ziel}
 $f$ normal $\Rightarrow$ $(V,h) $ besitzt eine ONB aus Eigenvektoren von $f$ (Spektralsatz)
\end{ziel}
\begin{bem}
	Sei  $(V,h)$ ein unitärer Raum, $f \in \operatorname{End}(V)$.
	Dann gilt:
	\begin{enumerate}[(a)]
		\item $U \subset V$ UVR mit $f(U) \subset U \Rightarrow f^{*}(U^{\bot}) \subset U^{\bot}$
		\item $f$ normal. Dann: $v \in V$ Eigenvektoren von $f$ zum Eigenwert $\lambda \in \C \Leftrightarrow v $ ist Eigenvektor von $f^{*}$ zum Eigenwert $\overline{\lambda}$
		\item $f$ selbstadjungiert $\Rightarrow$ Alle Eigenwerte von $f$ sind reell. $h(f^{*}(v),u) = \overline{h(u,f^{*}(v))}  = \overline{h(\underbrace{f(u),v)}}_{\in U} = 0 $
		$\Rightarrow f^{*}(v) \in U ^{\bot}$
		\item Sei $f$ normal. Setze $g := \lambda id_V - f$
	\end{enumerate}
\end{bem}
\begin{proof}
	\begin{enumerate}
		\item Sei $v\in V^{\bot}, u \in U$ es ist 
		\begin{enumerate}
			\item Beh.: $g^{*} = \overline{\lambda}id_V-f^{*}$
			\begin{align*}
				\text{ denn: } h((\lambda id_V-f)(x),y) 
				&= \lambda h(,y)-h(f(x),y) = h(x, \overline{\lambda}y)-h(x,f^{*}(y))\\
				&h(x,\overline{\lambda}y-f^{*}(y)) = h(x,(\overline{\lambda}id_V-f^{*}(y))) \text{ für alle } x,y \in V\\
			\end{align*}
				\item Beh.: $g^{*}\circ g=g\circ g^{*}$, d.h. $g$ ist normal 
				$
				\text{denn: } g \circ g^{*} = (\lambda id_V-f)\circ(\overline{\lambda}id_V-f^){*}f\circ f^{*} = f^{*}\circ f\underset{f \text{ normal}}{=}(\overline{\lambda}id_V-f^{*}\circ(\lambda id_V-f))= g^{*}\circ g$
				\item Sei $v\in V, v \neq 0$
		\begin{align*}
			\text{Dann: } 
			&v \text{ Eigenvektor zum Eigenwert } \lambda \text { von }f\\
			&v \text{ Eigenvektoren zum Eigenwert } \overline{\lambda}v\\
		\end{align*}
		
\item Sei $f$ selbstadjungiert, $\lambda \in \C$ ein Eigenwert von $f$, $v$ Eigenvektor zum Eigenwert $\lambda$ 
$\Rightarrow f$ normal, nach (b) ist $v$ Eigenvektor zum Eigenwert $\overline{\lambda}$ von $f^{*}=f \Rightarrow \lambda = \overline{\lambda}\Rightarrow \lambda \in \R$
\end{enumerate}
\end{enumerate}
\end{proof}
\begin{satz}[Spektralsatz für normale Operatoren]
Sei $(V,h)$ ein unitärer Raum, $f\in \operatorname{End}(V)$\newline
normal. Dann exisitiert eine ONB von $(V,h)$ aus Eigenvektoren von $f$.
\end{satz}
\begin{proof}
Beweis per Induktion nach $n=\operatorname{dim}V$.
\begin{itemize}
\item$n=0,1$: trivial
\item $n>1$: charakteristisches Polynom $\chi_f\in \C[t]$ hat nach dem Fundamentalsatz der Algebra eine Nullstelle in $\C$.\\
$\Rightarrow f$ hat einen Eigenwert, etwa $\lambda$.
Sei $v\in V$ ein Eigenvektor zu $\lambda$ mit $||v||=1$. Setze $L := \C v$. Es ist $f^{*}(v) = \overline{\lambda}v$, also $f^{*}(L) \subset L \overset{0.26 \text{(a)}}{\Rightarrow} \underset{=f}{(f^{*})^{*}}L^{\bot}\subset L^{\bot} \Rightarrow f$ induziert einen normalen Endimorphismus des unitären Raums $(L^{\bot},h\mid_{L^{\bot}\times L^{\bot}})$ Nach Induktionsvorraussetzung existiert eine ONB $(v_2,...,v_n)$ von $L^{\bot}$ aus Eigenvektoren zu $f\mid_{L^{\bot}}
	\Rightarrow (v,v_2,...,v_n) $ ist ONB von $V= L \hat{\oplus}L^{\bot} $ aus Eigenvektoren von $f$.
	\end{itemize}
    \end{proof}
    \begin{anm}
\begin{itemize}
	\item Es gilt sogar die Umkehrung: Wenn ONB von $(V,h)$ aus Eigenvektoren von $f$ exisitiert, dann ist $f$ normal.
	\item Für jeden selbstadjungierten/unitären Endomorphismus eines unitären Vektorraums existiert eine ONB von $(V,h)$ aus Eigenvektoren.
\end{itemize}
\end{anm}
\begin{lemma}
	Sei $A\in M_{n,n}(\C)$ normal. Dann existiert eine unitäre Matrix $U\in U(n)$, so dass $U^{*}AU$ eine Diagonalmatrix ist.
	\end{lemma}
\begin{proof}
	Wende Spektralsatz 0.27 auf $(\C^n, \langle \cdot, \cdot \rangle) \longrightarrow (\C^n, \langle \cdot, \cdot \rangle), x \mapsto Ax$ an. (Basiswechselmatrix unitär, da ONB von Eigenvektoren). Erhalte $U\in \operatorname{n,\C}$ mit $U^{-1}AU$ Diagonalmatrix, $U^{-1} = U^{*} $wegen $U$ unitär. 	
    \end{proof}
    \begin{anm}
Jede reelle orthogonale Matrix ist über $\C$ diagonalisierbar (aber: Es gibt orthogonale Matrizen, die über $\R$ nicht diagonalisierbar sind, z.B. $\operatorname{det}\left(\begin{matrix}
	{0}&{ -1 }\\
	{ 1 }&{0  }\\
	\end{matrix}\right)$
(Drehung um $\frac{\pi}{2}$)
\end{anm}
\newpage
\part{Ringe}
\section{Ringe und Ideale}
Erinnerung an LA 1 Definition:
\begin{definition}
	Ein \textbf{Ring} ist ein Tupel $(R,+,\cdot,0_R)$ bestehend aus einer Menge $R$ mit zwei Verknüpfungen 
	$$+,\cdot:R\times R\rightarrow R$$
	und einem ausgezeichnetem Element $0_R$, so dass gilt:
	\begin{enumerate}[(R1)]
		\item $(R,+,0_R)$ ist eine abelsche Gruppe 
		\item Assoziativität der Multiplikation: $a\cdot(b\cdot c)=(a\cdot b)\cdot c$ für alle $a,b,c\in R$
		\item Distributivität: $a(b+c)=ab+ac$, $(a+b)\cdot c=ac+bc$ für alle $a,b,c\in R$
	\end{enumerate}
	Ein \textbf{Ring mit Eins (Unitärer Ring)} ist ein Ring, in dem ein Element $1_R$ existiert, für das gilt
	\begin{enumerate}[(R4)]
		\item $1_R\cdot a = a = a \cdot 1_R$ für alle $a \in R$
	\end{enumerate}
	Ein Ring heißt \textbf{kommutativ}, wenn die Multiplikation kommutativ ist, d.h. heißt wenn gilt:
	\begin{enumerate}[(R5)]
		\item $a\cdot b = b \cdot a$ für alle $a,b\in R$
	\end{enumerate}
\end{definition}
\textbf{Konvention}: In der LA2 interessieren wir uns für kommutative Ringe mit eins. Deswegen verwenden wir ab jetzt folgende Sprechweise:
\textbf{Ring:=Kommutativer Ring mit Eins}
\begin{bsp} Beispiele für Ringe:
	\begin{itemize} 
		\item $\Z,\Q,\R,\C$
		\item $\Z/n\Z$
		\item Nullring: $\{0\}$. Hierbei $0_R=0=1_R$. Häufig schreibt man kurz 0 für den Nullring.
	\end{itemize}
\end{bsp}
In diesem Abschnitt seien $R$ und $S$ stets Ringe.
\begin{definition}
	Sei $J\subseteq R$. $J$ heißt \textbf{Ideal} in $R \overset{\text{Def:}}{\Leftrightarrow}$ Die folgenden Bedingungen sind erfüllt:
	\begin{enumerate}[(J1)]
		\item  $0\in J$
		\item $a,b \in J \implies a+b\in J$
		\item $r\in R, a \in J \implies ra \in J$
	\end{enumerate}
\end{definition}
\begin{bsp}
	\begin{enumerate}[(a)]
		\item $\{0\}, R$ sind Ideale in in $R$.
		\item  Für $n \in \Z$ ist $nZ:=\{na|a\in \Z\}$ ist ein Ideal in $\Z$
	\end{enumerate}
\end{bsp}
\begin{ziel}
Jedes Ideal in $\Z$ ist von der Form  $n\Z$.
\end{ziel}
\begin{bem}[Division mit Rest]
	Seien $a,b\in\Z,b\neq 0$. Dann existieren $q,r\in\Z$ mit 
	$$a=qb+r \text{ und } 0\leq r\leq |b|$$
	$r$ heißt \textbf{Rest} der Division von $a$ durch $b$.
\end{bem}
\begin{proof}
	Setze $R:=\{a-\overset{\sim}{q}b|\overset{\sim}{q}\in\Z\}\cap \N_0$
	$\implies R$ ist nichtleere Teilmenge von $N_0$, insbesondere besitzt $R$ kleinstes Element, etwa $r$.
	Sei $q\in\Z$ mit $a-qb=r \implies a=qb+r$
	\textit{Annahme}: $r\geq |b|\implies 0\leq r-|b|=a-qb-\operatorname{sgn}(b)b=\underbrace{a-(q+\operatorname{sgn}(b))}_{\in R}b<r$ Das ist ein Widerspruch zur Minimalität von $r$.
\end{proof}
\begin{anm}
 $q,r$ wie in Bemerkung 1.5 sind eindeutig bestimmt.
\end{anm}
\begin{bem}
	Sei $J \subseteq \Z$ ein Ideal. Dann existiert ein $n\in\Z$ mit  $J=n\Z$
\end{bem}
\begin{proof}
	\begin{itemize}
	\item Falls $J=\{0\}=0\Z$, dann fertig.
	\item Im Folgenden sei $J\neq \{0\}$. Dann existiert ein Element $a\in J, a\neq 0$. Mit $a\in J$ ist auch $(-1)a=-a\in J$, somit $J \cap \N\neq \varnothing \implies  J\cap\N$ besitzt ein kleinstes Element, etwa $n$.
	\textit{Behauptung}: $J=n\Z$
		\begin{enumerate}[(i)]
			\item "$\supseteq$": Sei $x\in n\Z \implies $ Es existiert ein $q\in \Z$ mit $x=\underbrace{nq}_{\in J}\overset{\text{J Ideal}}{\implies}x\in J$
			\item "$\subseteq$": Sei $x\in J \overset{\text{Division mit Rest}}{\implies}$ Es existieren $q,r \in \Z$ mit x=qn+r, $0\leq r < n \implies r = \underbrace{n}_{\in J}-\underbrace{qn}_{\in J}\in J$. Wegen der Minimalität von  $n$ in $J \cap \N$ folgt $r=0 \implies x=qn\in \Z$
		\end{enumerate}
	\end{itemize}
\end{proof}
\begin{definition}
	Sei $\varphi: R \longrightarrow S$ eine Abbildung. $\varphi$ heißt ein \textbf{Ringhomomorphismus} $\overset{\text{Def:}}{\implies}$ Die folgenen Bedingungen sind erfüllt:
	\begin{enumerate}[(RH1)]
		\item $\varphi(a+b)=\varphi(a)+\varphi(b)$ für alle $a,b \in R$
		\item $\varphi(ab)=\varphi(a)\varphi(b)$ für alle $a,b\in R$
		\item $\varphi(1_R)=1_S$
	\end{enumerate}
\end{definition}
\begin{bem}
	Sei $\varphi:R\longrightarrow S$ ein Ringhomomorphismus. Dann gilt:
		\begin{enumerate}[(a)]
			\item $J\subseteq S$ Ideal $\implies (\varphi)^{-1}(J)\subseteq R$ Ideal
			\item $\ker \varphi:=\{a\in R |  \varphi(a)=0\}\subseteq R$ Ideal
			\item $\varphi$ injektiv $\Leftrightarrow$ $\ker\varphi = \{0\}$
			\item $J\subseteq R$ Ideal und $\varphi$ surjektiv $\implies \varphi(S)\subseteq S$ Ideal
			\item $\operatorname{im} \varphi:=\varphi(R)$ ist ein Unterring von $S$
		\end{enumerate}
\end{bem}
\begin{proof}
	\begin{enumerate}[(a)]
		\item 
		\begin{enumerate}[(J1)]
			\item $0\in\varphi^{-1}(J)$, denn: $\varphi(0)=\varphi(0+0)=\varphi(0)+\varphi(0)\implies\varphi(0)=0\in J$\\$\implies 0\in \varphi^{-1}(J)$
			\item $a,b\in\varphi^{-1}(J)\implies \varphi(a),\varphi(b)\in J\overset{\text{J Ideal}}{\implies} \underbrace{\varphi(a)+\varphi(b)}_{=\varphi(a+b)}\in J\implies a+b\in\varphi^{-1}(J)$
			\item $r\in r, a \in \varphi^{-1}(J)\implies\varphi(a)\in J\overset{\text{J Ideal}}{\implies}\underbrace{\varphi(r)\varphi(a)}_{=\varphi(ra)}\in J \implies ra\in \varphi^{-1}(J)$
		\end{enumerate}
		\item aus (a) wegen $\ker \varphi=\varphi^{-1}(\{0\})$,$\{0\}\subseteq S$ Ideal.
		\item nachrechnen
		\item nachrechnen
		\item nachrechnen
	\end{enumerate}
\end{proof}
\begin{anm}
 (d) wird falsch, wenn man die Vorraussetzung $\varphi$ surjektiv weglässt. Die kanonische Inklusion $i.\Z\longrightarrow\Q, x\longmapsto x$ ist ein Ringhomomorphismus, $\Z$ ein Ideal in $\Z$, $\Z=i(\Z)$ ist kein Ideal in$\Q$(denn: $\frac{1}{3}\cdot 2 = \frac{2}{3}\notin\Z$). $\Z$ ist aber ein Unterring in $\Q$.
\end{anm}
 \begin{bem}
	Sei $J\subseteq R$ ein Ideal. Dann ist durch $r_1\sim r_2 \overset{\text{Def:}}{\Leftrightarrow}r_1-r_2\in J$ eine Äquivalenzrelation auf $R$, welche die zusätzliche Eigenschaft
	$$r_1\sim r_2 , s_1\sim s_2 \implies r_1+s_1 \sim r_2 + s_2, r_1s_1\sim r_2s_2$$
	(Kongruenzrelation) hat, definiert. Die Äquivalenzklasse von $r\in R$ ist durch 
	$$\overline{r}:=r+J:=\{r+a|a\in J\}$$
	gegeben und heißt die \textbf{Restklasse} von $r$ modulo $J$. Die Menge der Resklassen bezeichnen wir mit $R/J$.
\end{bem}
\begin{proof}
	\begin{enumerate}[(1.)]
		\item "$\sim$" ist eine Äquivalenzrelation:
		\begin{itemize}
			\item $ \sim$ reflexiv: $ r\sim r$, denn $r-r=0\in J$
			\item $\sim$ symmetrisch: Seien $r,s\in R$ mit $r\sim s \implies r-s \in J \implies (-1)(r-s)\in J \implies s \sim r \in J$
			\item $\sim$ transitiv: Seien $r,s,t\in R $ mit $r\sim s, s \sim t \implies r-s\in J, s-t\in J\implies r-t\in J\implies r-t=(r-s)+(s-t)\in J\implies r \sim t$
		\end{itemize}
	\item Verträglichkeit mit $+,\cdot:$ Sei $r_1\sim r_2. s_1\sim s_2 \implies r_1-r_2\in J, s_1-s_2\in J$
			$$(r_1+s_1)-(r_2+s_2)=\underbrace{(r_1-r_2)}_{\in J}+\underbrace{(s_1+s_2)}_{\in J}	\implies r_1+s_1\sim r_2+s_2$$
			Außerdem:
			$$r_1s_1-r_2s_2=r_1\underbrace{(s_1-s_2)}_{\in J}+s_2\underbrace{(r_1-r_2)}_{\in J} \implies r_1s_1\sim r_2s_2$$
	\end{enumerate}
\end{proof}
\begin{bem}
	Sei $J \subseteq R$ ein Ideal. Dann wird $R/J$ mit der Addition 
	$$+: R/J \times R/J \longrightarrow R/J, \overline{r}+\overline{s}:= \overline{r+s}$$
	und der Multipikation 
	$$\cdot: R/J\times R/J \longrightarrow R/J, \overline{r}\cdot \overline{s}:=\overline{r\cdot s}$$
	zu einem Ring, dem \textbf{Faktorring (Restklassenring)} $R/J$. Die Abbildung $\pi:R \longrightarrow R/J, r\mapsto\overline{r}$ ist ein surjektiver Ringhomomorphismus mit $\operatorname{ker}\pi=J$.
	$\pi$ heißt die  \textbf{kanonische Projektion} von $R$ nach $R/J$.
\end{bem}
\begin{proof}
	\begin{itemize}
		\item Wohldefiniertheit von $+,\cdot$: nach 1.9 ist für $r_1,r_2s_1,s_2\in R$ mit $r_1 \sim r_2, s_1\sim s_2$ auch $r_1+s_1 \sim r_2+s_2, r_1s_1\sim r_2s_2$
		\item Ringeigenschaften vererben sich aufgrund der vertreterweisen Definition
		\item $\pi$ ist ein Ringhomomorphismus nach Konstruktion: $\pi(a+b)=\overline{a+b}=\overline{a}+\overline{b}=\pi(a)+\pi(b)$, analog für $\cdot$, $\pi(1)=\overline{1}$
		\item $\pi$ ist surjektiv nach Konstruktion 
		\item $\operatorname{ker}\pi=\{r\in R|\overline{r}=\overline{0}\}=\{r\in R|r\sim 0\}=\{r \in R|r-0\in J\}=J$
	\end{itemize}
\end{proof}
\begin{anm}
 Insbesondere sind die Ideale in $R$ genau die Kerne von Ringhomomorphismen, die von $R$ ausgehen.
\end{anm}
 \begin{bsp}
	Ist $R=\Z,J=n\Z$, dann erhält man die aus der LA1 bekannten Restklassenringe:
	$\Z/n\Z=\{\overline{0},...,\overline{n-1}\}$ mit den Verknüpfungen $\overline{a}+\overline{b}:=\overline{a+b},\overline{a}\cdot\overline{b}=\overline{a\cdot b}$.
\end{bsp}
\begin{satz}[Homomorphiesatz für Ringhomomorphismen]
	Sei $\varphi:R \longrightarrow S$ ein Ringhomomorphismus. Dann gibt es einen Ringhomomorphismus 
	$$\phi: R/\operatorname{ker \varphi} \longrightarrow \operatorname{im}\varphi, \overline{r}=r+\operatorname{ker}\varphi\mapsto \varphi(r).$$
\end{satz}
\begin{proof}
	\begin{enumerate}
		\item Wohldefiniertheit von $\phi$: Seien $r_1,r_2\in R$ mit $\overline{r_1}=\overline{r_2}\implies r_1-r_2\in \operatorname{ker}\varphi\implies \varphi(r_1-r_2)=0\implies \varphi(r_1)=\varphi(r_2)$
		\item $\phi$ ist ein Ringhomomorphismus: $\phi(\overline{r_1}+\overline{r_2})=\phi(\overline{r_1+r_2})=\varphi(r_1+r_2)=\varphi(r_1)+\varphi(r_2)=\phi(\overline{r_1})+\phi(\overline{r_2})$, analog für "$\cdot$", $\phi(1)=\varphi(1)=\overline{1}$
		\item $\phi $ ist injektiv: Sei  $r\in R$ mit $\phi(\overline(r))=0\implies\varphi(r)=0\implies r\in \operatorname{ker}\varphi\implies r-0\in \operatorname{ker}\varphi\implies \overline{r}=\overline{0}$, d.h. $\operatorname{ker}\phi=\{\overline{0}\}$
		\item $\phi$ ist surjektiv: Nach Konstruktion
	\end{enumerate}
\end{proof}
\begin{bsp}
	Seien $K$ ein Körper, $R=K[t]$, $\varphi:K[t]\longrightarrow K, f \mapsto f(0)$. 
	$\varphi$ ist ein Ringhomomorphismus, $\operatorname{im}\varphi= K, \operatorname{ker}\varphi=\{f\in K[t]|f(0)=0\}=\{tg|g\in K[t]\}=tK[t]$. Wir erhalten einen Ringhomomorphismus 
	$$	\phi:K[t]/tK[t] \overset{\cong}{\longrightarrow} K, f+tK[t]\mapsto f(0)$$
\end{bsp}
\begin{bem}
	Seien $J \subseteq R$ ein Ideal , $\pi: R \longrightarrow R/J$ die kanonische Projektion. Dann sind die Abbildungen
	\begin{align*}
	\{\text{Ideale in }R/J\} \overset{\longrightarrow}{\longleftarrow}&\{\text{ Ideale }\overset{\sim}{J}\text{ in }R \text{ mit } \overset{\sim}{J}\supseteq J\}\\
	J\longmapsto&\pi^{-1}(J)\\
	J\longmapsto& \pi(J)
	\end{align*}
	zueienander inverse, inklusionserhaltende Abbildungen.
\end{bem}
\begin{proof}
	\textit{Übung}
\end{proof}
\begin{definition}
	$x\in R$ heißt eine \textbf{Einheit} $\overset{\text{Def:}}{\Leftrightarrow}$ Es existiert ein  $y\in R$ mit $xy=1_R$.
	$R^{\times}:=\{x\in R| x \text{ ist Einheit }\}$ bildet eine abelsche Gruppe bzgl "$\cdot$", die \textbf{Einheitengruppe} von R.
\end{definition}
\begin{anm}
\begin{itemize}
	\item vgl. LA1 Lemma 1.11 
	\item $R$ ist Körper $\Leftrightarrow R^{\times}=R\setminus \{0\}$
	\item häufig wird die alternative Notation $R^{*}$ statt $R^{\times}$ benutzt.
\end{itemize}
\end{anm}
\begin{bsp}
	\begin{enumerate}[(a)]
		\item $\Z^{\times}=\{-1,1\}$, denn $1\cdot 1= 1$ und $(-1)(-1)=1$
		Sind $a,b\in \Z$ und $ab=1 \implies a=b=1$ oder $a=b=-1$
		\item $K$ Körper , $(K[t])^{\times}=K^{\times}$
	\end{enumerate}
\end{bsp}
\newpage
\begin{bem}
Sei $R\neq 0$. Dann sind äquivalent:
\begin{enumerate}[(i)]
	\item $R$ ist ein Körper 
	\item \{0\} und $R$ sind die einzigen Ideale in $R$ 
	\item Jeder Ringhomomorphismus $\varphi: R \longrightarrow S$ in einen Ringhomomorphismus $S\neq 0$ ist injektiv
\end{enumerate}
\end{bem}
\begin{proof}
	\begin{itemize}
		\item (i) $\implies $(ii) Sei $R$ ein Körper. Sei $J\subseteq R$ ein Ieal, $J\neq \{0\}$. Es exisitiert ein $a\in J, a\neq 0\implies 1=\underbrace{a}_{\in J}a^{-1}\in J\implies $ ist 	$b\in R$, dann ist $b=b \cdot \underbrace{1}_{\in J}\in J$, d.h.  $J=R$
		\item (ii)$\implies$(iii) Sei $\varphi: R \longrightarrow S$ ein Ringhomomorphismus mit $S\neq 0$. Nach 1.8 (a) ist $\operatorname{ker}\varphi\subseteq R$ ein Ideal, d.h. wegen (ii) ist $\operatorname{ker}\varphi=\{0\}$ oder $\operatorname{ker}\varphi = R$.
		Es ist $\operatorname{ker}\varphi\neq R$, denn $\varphi(1_R)=1_S$ und $1_S\neq 0_S$ (Wäre $1_S=0_S$, dann ist für Jedes $a\in S$:$a=a\cdot 1_S=a\cdot 0_S=0_S$, d.h. $S=0$ \textit{Widerspuch})$\implies \operatorname{ker}\varphi=\{0\}$, d.h. $\varphi$ ist injektiv.
		\item (iii) $\implies$ (i) Sei $a\in R\textbackslash R^{\times}$, insbesondere existiert kein $b\in R$ mit $ab=1_R\implies aR:=\{ar|r\in R\}\subsetneq R$, und $aR$ ist ein Ideal in $R$.
		$\implies R/aR$ ist nicht der Nullring (denn: Wenn $R/aR=0$, dann $1_R+aR=0_R+aR$, also $1\in aR$ \textit{Widerspruch})$\overset{\text{(iii)}}{\implies}$ Die kankonische Projektion $\pi: R \longrightarrow S=R/aR$ ist injektiv, d.h. $\operatorname{ker}\pi = \{0\}$, anderer seits ist $\operatorname{ker}\pi=aR$ nach 1.10, also: $\underbrace{a\cdot 1_R}_{\in aR}=\{0\}\implies a=0$, d.h. $R$ ist Körper.
	\end{itemize}
\end{proof}
\begin{definition} 
	$x\in R$ heißt \textbf{Nullteiler} $\overset{\text{Def:}}{\Leftrightarrow}$ Es existiert ein $y\in R$, $y\neq 0_R$ mit $xy=0_R$.
	R heißt \textbf{nullteilerfrei} $\Leftrightarrow R\neq 0$ und $0\in R$ ist der einzige Nullteiler in $R$ (\textbf{Integritätsbereich}).
\end{definition}
\begin{anm}
\begin{itemize}
	\item $R\neq 0 \implies 0_R$ ist ein Nullteiler in $R$ (wegen $0_R\cdot 1_R=0_R, 0_R\neq 1_R$)(Achtung: Unterschiedliche Notatio in Literatur)
	\item Im Nullring ist $0$ kein Nullteiler (aber: Nullring ist nicht nullteilerfrei)
\end{itemize}
\end{anm}
\begin{bsp}
	\begin{enumerate}[(a)]
		\item $\Z$ ist nullteilerfrei
		\item $\overline{2}\in \Z/6\Z$ ist Nullteiler wegen $\overline{2}\cdot \underbrace{\overline{3}}{\neq 6}= \overline{0}$ in $\Z/6\Z$
		\item Sei $K$ Körper, dann ist $K[t]$ nullteilerfrei
	\end{enumerate}
\end{bsp}
\begin{definition}
Seien $a_1,...,a_n\in R$,$J\subseteq R$ ein Ideal. 
$$	(a_1,...,a_n):=\{\sum_{i=1}^{n}a_ir_i|r_1,...,r_n\in R\} \subseteq R \text{ heißt das \textbf{ von $a_1,...,a_n$ erzeugte Ideal}}$$
\begin{align*}
J& \text{ heißt \textbf{Hauptideal} }\overset{\text{Def:}}{\Leftrightarrow}\text{ Es existiert } a\in R \text{ mit } J=(a)=\{ra|r\in R\} =: Ra (=aR).\\
R&\text{ heißt ein \textbf{Hauptidealring} (HIR)}\overset{\text{Def:}}{\Leftrightarrow}R\text{ ist nullteilerfrei und jedes Ideal in } R \text{ ist ein Hauptideal.}
\end{align*}
\end{definition}
\begin{anm}
 $(a_1,..,a_n)$ ist ein Ideal in $R$ (leicht nachzurechnen)
\end{anm}
\newpage
\begin{bsp}
	\begin{enumerate}[(a)]
		\item $K$ Körper $\implies K$ ist HIR (denn: $K$ Körper $\overset{1.17}{\implies}\{0\}, R$ sind dei einzigen Ideale in $R$, $\{0\}=(0),R=(1)=\{1\cdot r|r\in R\}$ und $K$ ist nullteilerfrei (vgl LA1, Lemma 1.15))
		\item $\Z$ ist ein HIR, denn: $\Z$ ist nullteilerfrei und jedes Ideal in $\Z$ ist von der Form $n\Z=(n)$ (das ist Bemerkung 1.6)
		\item $\Z[t]$ ist kein HIR, denn: Es gibt kein $f\in \Z[t]$ mit $(2,t)=(f)$
		\begin{proof}
			Annahme: Es existiert ein $f\in \Z[t]$ mit $(f)=(2,t)$, dann existiert $h\in\Z[t]$ mit $z=hf\implies \operatorname{deg}h=\operatorname{deg}f=0$, d.h. $f$ ist konstant (?), etwa $f=a$ für ein $a\in \Z$. Außerdem existiert ein $\overset{\sim}{h}\in \Z[t]$ mit $t=\overset{\sim}{h}f=\overset{\sim}{h}a\overset{\text{t normiert}}{\implies}a=±1\implies f=±1$, aber: $±1\notin (2,t)$, denn anderenfalls existieren $u,v\in \Z[t]$ mit $±1=2\cdot u+t\cdot v\overset{t=0}{\implies }±1=2\cdot u(0)+0\cdot v(0)=	2\cdot 1(0)$ \textit{Widersrpuch}
		\end{proof}
	\end{enumerate}
	\end{bsp}
	\begin{definition}
		Sei $J\subseteq R$ ein Ideal.
		$J$ heißt 
		\begin{align*}
		\text{\textbf{Primideal}} \overset{\text{Def:}}{\Leftrightarrow}&J \neq R \text{ und für alle } x,y\in R \text{ gilt: } xy\in J\implies x\in J \text{ oder } y\in J.\\
		\text{\textbf{maximales Ideal}}\overset{\text{Def:}}{\Leftrightarrow}& J\neq R \text{ und es existiert kein Ideal } I\subseteq R\text{ mit }J\subsetneq I \subsetneq R\\
		\end{align*}
		(d.h. $J$ ist maximal bezüglich "$\subseteq$" unter allen Idealen $\neq R$ in $R)$
	\end{definition}
	\begin{bem}
		Sei $J\subseteq R$ ein Ideal. Dann gilt:
		\begin{enumerate}[(a)]
			\item $J$ ist Primideal $\Leftrightarrow R/J$ nullteilerfrei
			\item $J$ maximales Ideal $\Leftrightarrow$ $R/J$ Körper 
		\end{enumerate}
	\end{bem}
	\begin{proof}
		\begin{enumerate}[(a)]
			\item Die Bedingung $xy \in J \implies x\in J$ oder $y\in J$ ist äquivalent zu $\overline{x}\cdot\overline{y}=\overline{0}\implies \overline{x}=\overline{0}$ oder $\overline{y}=\overline{0}$ in $R/J$ $J\neq R$ ist äquivalent zu $R/J\neq 0$. D.h. $J$ Primideal ist äquivalent zur Nullteilerfreiheit von $R/J$.
			\item Bemerkung 1.16: Ideale $I\subseteq R$ mit $J\subsetneq I \subsetneq R$ entsprechen genau den Idealen in $R/J$, die $\neq \{0\}$ und $\neq R/J$ sind. Nach Bemerkung 1.17 ist $R/J$ genau dann ein Körper, wenn es solche Ideale nicht gibt.
		\end{enumerate}
    \end{proof}
\begin{fg}
Sei $J\subseteq R$ ein maximales Ideal. Dann ist $J$ ein Primideal:
\end{fg}
	\begin{proof} 
		Folgt aus 1.23, da jeder Körper nullteilerfrei ist (LA1, Lemma 1,15)
    \end{proof}
    \begin{frage}
 Primideale/maximale Ideale in $\Z$?
    \end{frage}
\begin{bem}
	Sei $n\in \N$. Dann sind äquivalent:
	\begin{enumerate}[(i)]
		\item $n$ ist Primzahl
		\item $\Z/n\Z$ ist nullteilerfrei
		\item $\Z/n\Z$ ist ein Körper
	\end{enumerate}
\end{bem}
\newpage
\begin{proof}
	\begin{itemize}
		\item (i) $\Leftrightarrow$ (iii): LA1, Lemma 1.16, Bemerkung 1.17
		\item (iii) $\implies$ (ii): Körper sind nullteilerfrei. LA 1; Lemma 1.15
		\item (ii) $\implies$ (i): Beweis durch vollständige Induktion:
		\begin{enumerate}
			\item Falls $n=1$, dann $\Z/n\Z=\Z/\Z=0$ nicht nullteilerfrei
			\item Falls $n>1$, Keine Primzahl, dann $n=ab$ mit $1<a,b<n	\implies \overline{0}=\overline{n}=\overline{a}\cdot\overline{b}\implies \Z/n\Z$ nicht nullteilerfrei.
		\end{enumerate}
	\end{itemize}
\end{proof}
\begin{fg}
\begin{itemize}
	\item Primideale in $\Z$:$(0),(p)$ für $p$ Primzahl.
	\item Maximale Ideale in $\Z$: $(p)$ für $p$ Primzahl 
\end{itemize}
\end{fg}
\begin{proof}
	Für $n<0$ ist $(-n)=(n)$. Rest aus 1.25
\end{proof}
\begin{ziel}
 Jeder Ring $\neq 0$ hat ein maximales Ideal.
\end{ziel}
\begin{anm}
    Dafür bwnötigen wir ein Axiom aus der Mengenlehre, das \textbf{Auswahlaxiom}. Ist $I$ eine Menge und $(A_i)_{i\in J}$ ein Familie von nichtleeren Mengen, dann gibt es eine Abbildung $$\gamma: I \longrightarrow \bigcup\limits_{i\in J}(A_i) \text{ mit } \gamma(i)\in A_i, \forall i \in I\text{ (Auswahlfunktion)}$$
    Das Auswahlaxiom ist äquivalent zu folgenden Aussagen:
    \begin{itemize}
        \item Zornsches Lemma (1.32)
        \item Jeder Vektorraum hat eine Basis 
        \item Jeder Ring $\neq = 0$ hat ein maximales Ideal.
    \end{itemize}
\end{anm}
\begin{definition}
    Sei $M$ eine Menge, $\sim$ eine Relation auf $M$.
    \begin{itemize}
        \item \begin{align*} \sim \text{ heißt \textbf{antisymmetrisch}}&\overset{\text{Def:}}\Leftrightarrow \text{ Für alle }a,b\in M\text{ gilt }: a\sim b \text{ und }b\sim a \implies a=b\\
        \text{\textbf{total}}&\overset{\text{Def:}}\Leftrightarrow \text{ Für alle }a,b\in M \text{ gilt }: a \sim b \text{ oder } b\sim a
        \end{align*}
        \item  \begin{align*} \sim \text{ heißt \textbf{ Halbordnung} auf }M&\overset{\text{Def:}}\Leftrightarrow \sim \text{ reflexiv, antisymmetrisch und transitiv}\\
        \text{\textbf{Totalordnung} auf }M&\overset{\text{Def:}}\Leftrightarrow \sim \text{ ist eine Halbordnung und }\sim \text{ ist total}
        \end{align*}
    \end{itemize}
    In diesen Fällen sagt man auch: Das Tupel $(M,\sim)$ ist eine halbgeordnete bzw. totalgeordnete Menge. 
\end{definition}
\begin{bsp}
    \begin{enumerate}[(a)]
        \item $\leq$ ist auf $\N$ eine Totalordnung 
        \item Sei $M=\mathbb{P}(\{1,2,3\})$, $\subseteq$ auf $M$ eine Halbordnung, aber keine Totalordnung. Es ist zum Beispiel weder $\{1\}\subset\{3\}$ noch $\{3\}\subset\{1\}$.
    \end{enumerate}
\end{bsp}
\begin{definition}
    Sei $(M\leq)$ eine halbgeordnete Menge, $a\in M$. a heißt ein \textbf{maximales Element} vom $M \overset{\text{Def:}}\Leftrightarrow$ Für alle $x\in M$ gilt $a\leq x \implies x=a$
\end{definition}
\begin{anm}
    Für ein maximales Element $a\in M$ gilt nicht notwendig $x\leq a$ für $x\in M$. Im allgemeinen extisieren maximale Elemente nicht unbedingt.
\end{anm}
\begin{bsp}
    \begin{enumerate}[(a)]
        \item  In $(\{\{1\},\{2\},\{3\},\{1,2\},\{2,3\},\{1,3\},\subseteq\})$ sind $\{1,2\},\{2,3\},\{1,3\}$ maximale Elemente.
        \item maximale Ideale im Ring $R$ sind maximale Elemente von $\{I\not\subseteq R|I\text{ ist Ideal}\}$ bezüglich $\subseteq$.
    \end{enumerate}
\end{bsp}
\begin{definition}
    Sei $(M,\leq)$ eine halbgeordnete Menge. $(M,\leq)$ heißt \textbf{induktiv geordnet}$\overset{\text{Def:}}\Leftrightarrow$ Jede Teilmenge von $T\in M$, für die $(T,\leq)$ totalgeordnet ist, besitzt eine obere Schranke, d.h. es existiert ein $S\in M$ mit $t\leq S$ für alle $t\in T$.
\end{definition}
\begin{satz}[Zornsches Lemma]
    Jede induktiv geordnete nichtleere Menge $(M,\leq)$ besitzt ein maximales Element.
\end{satz}
\begin{anm}
    Das zornsche Lemma ist äquivalent zum Auswahlaxiom.
\end{anm}
\begin{satz}
    Sei $R\neq 0$. Dann besitzt $R$ ein maximales Ideal.
\end{satz}
\begin{proof}
    Sei $\mathrm{X}:=\{I\not\subseteq R| I \text{ Ideal}\}$
    \begin{itemize}
        \item $\mathrm{X}$ ist bzgl. $\subseteq$ halbgeordnet
        \item $\mathrm{X}\neq\emptyset$ wegen $\{0\}\in \mathrm{X}$
        \item Sei $\{I_\lambda|\lambda \in 1 \}$ totalgeordnete Teilmenge von $\mathrm{X}$ (d.h. für $\lambda,\mu \in 1:I_\lambda \subseteq I_\mu$ oder $I_\mu\subseteq I_\lambda$)
            Behauptung: $\{I_\lambda|\lambda\in 1\}$ besitzt eine obere Schranke in $\mathrm{X}$, d.h, es existiert ein $J\in \mathrm{X}$ mit $I_\lambda\subseteq I$ für alle $\lambda\in 1$
            denn: Setze $I:=\bigcap\limits_{\lambda\in 1}I_\lambda$
            \begin{enumerate}
                \item I ist ein Ideal, denn: $0\in I$ wegen $0\in I_\lambda$ für alle $\lambda\in 1$
                \begin{enumerate}[(J2)]
                    \item $a,b\in J \implies $ Es existiert $\lambda, \mu$  mit $a\in J_\lambda, b\in I_\mu,$ ohne Einschränkungen gelte $I_\lambda \subseteq I_\mu\implies \underbrace{a}_{\in I_\lambda\subseteq I_\mu}+\underbrace{b}_{\in I_\mu}\in I_\mu\subseteq I$ 
                    \item $a\in J, r\in R\implies $ Es existiert $\lambda\in 1$ mit $a\in I_\lambda \implies ra\in I_\lambda\subseteq I$
                \end{enumerate}
        \item $I\not\subseteq R$, denn $i\subseteq R$ und  $I\neq R$ wegen $1\neq I_\lambda$ für alle $\lambda\in 1$, (d.h. $I\in \mathrm(X)$)
        \item $I_\lambda\subset I$ für alle $\lambda\in 1$.
            \end{enumerate}
    Zornsches Lemma: $\mathrm(X)$ besitzt maximales Element $I$ bzgl $\subseteq \implies I$ ist maximales Ideal in $R$.
    \end{itemize}
\end{proof}
\begin{fg}
    Es gilt:
    \begin{enumerate}[(a)]
        \item Jedes Ideal $I\not\subseteq R$ ist einem Ideal von $R$ enthalten.
        \item Jedes $x\in R\setminus R^{\times} $ ist einem Ideal von $R$ enthalten.
    \end{enumerate}
\end{fg}
\begin{proof}
    \begin{enumerate}[(a)]
        \item $J\not\subseteq R$ Ideal $\implies R/I\neq 0$, also besitzt $R/I$ ein maximales Ideal $\overset{1.14}{\implies} R $ besitzt ein maximales Ideal, das $I$ enthält.
        \item Sei $x\in R\setminus R^{\times}\implies (x)\not\subseteq R$, denn $ 1\notin (x)$. Behauptung folgt aus (a)
    \end{enumerate}
\end{proof}
\begin{ziel}
    Formulierung und Beweis des chinesischen Restsatzes.
\end{ziel}
\begin{definition}
    Seien $I,J\subseteq R$ Ideale. Dann sind 
    $$I+J := \{a+b{a\in I, b\in J}\}$$
    $$I\cdot J := \{\sum_{i=1}^{n}a_ib_i|n\in \N_0, a_1,...,a_n\in I, b_1,..,b_n\in J\}$$
    und $I\cap J $ Ideale in $R$.
    Analog für endliche Familien von Idealen, insbsondere $I^n:=\underbrace{I\cdot ... \cdot I}_{n \text{-mal}}$ für $n\in \N$. Konvention: $I^0:=R$. I,J heißen \textbf{relativ prim}$\overset{\text{Def:}}\Leftrightarrow i+J =R=(1)$
\end{definition}
\begin{anm}
    \begin{itemize}
        \item Das dies tatsächlich Ideale sind, rechnet man nach
        \item Offenbar ist Multiplikation bzw. Addition von Idealen assoziaztiv, Klammerung ist nicht notwendig
        \item $(a_1,...,a_n) = (a_1)+...+(a_n)$
    \end{itemize}
\end{anm}
\begin{bsp}
    Seien $R=\Z, I=(2), J=(3)$
    \begin{itemize}
        \item $I+J=(1)$, denn: $ 1=\underbrace{(-1)\cdot 2}_{\in I}+\underbrace{1\cdot 3}_{\in J}\in I+J$
        \item $I \cap J = (6)$
        \item $IJ=(6)$
    \end{itemize}
\end{bsp}
\begin{anm}
     Für $R=\Z$ ist $(m)+(n) =(\operatorname{m,n}),(m)\cap(n)=(\operatorname{kgV}(m,n)),(m)(n)=(mn)$

\end{anm}
\begin{bem}
    $I,J,\subseteq R$ Ideale. Dann gilt:
    \begin{enumerate}[(a)]
        \item $I(J+K)=IJ+IK$
        \item $(I\cap J)(I+J)\subseteq IJ \subseteq I\cap J$
        \item $I+J=(1)\implies I\cap J = IJ $
    \end{enumerate}
\end{bem}
\begin{proof}
    Übung.
\end{proof}
\begin{bem}
    Seien $I_1,...,I_n\subseteq R$ paarweise relative Primideale. Dann gilt:
    $$I_1\cdot ... \cdot I_n=I_1\cap ... \cap I_n$$
\end{bem}
\begin{proof}
    Beweis durch Induktion nach n:
    \begin{itemize}
        \item $n=2$: aus 1.37 (c)
        \item $n\geq 3$: Behauptung sei wahr für alle $k<n$. Setze $J:=I-I_1\cdot...\cdot I_{n-1}\overset{IV}{=}I_1\cap...\cap I_{n-1}$
        Behauptung: $J+I_{n}=(1)$. Denn: Nach Vorraussetzung ist $I_{j}+I_n=(1)$ für $j=1,...,n-1$
        \begin{align*}
            &\implies \text{ Für alle } j\in \{1,...,n-1\} \text{ existieren } x_j\in I_j,y_j\in I_n \text{ mit } x_j+y_j=1\\
            &\implies x_1\cdot ... \cdot x_n-1 = (1-y_1)\cdot ... \cdot (1-y_{n-1})\\
            &\implies x_1\cdot ... \cdot x_{n-1} = 1+ y \text{ für ein } y \in I_n\\
            &\implies 1 = \underbrace{x_1\cdot ... \cdot x_{n-1}}_{\in I_1\cdot...\cdot I_{n-1}=J}+\underbrace{(-1)y}_{I_n}\in J+I_n, \text{ d.h. } J+In = (1)
        \end{align*}
        Somit: $I_1\cdot ... \cdot I_n = J \cdot I_n =J\cap I_n = (I_1\cap ... \cap I_{n-1})\cap I_n=I_1\cap ... \cap I_n.$
    \end{itemize}
    \end{proof}
\begin{definition}
    Sei $(R_i)_{i\in I}$ eine Familie von Ringen. Das kartesische Produkt $\prod_{i\in I}R_i$ wird durch komponentenweise Addition und Multiplikation zu einem Ring. Diesen bezeichnet man als das \textbf{direkte Produkt} über die Familie $(R_i)_{i\in I}.$
\end{definition} 
\begin{satz}[Chinesischer Restsatz]
    Seiene $I_1,...,I_n \in R $ Ideale, $\varphi : R \longrightarrow \prod_{j=1}^{n}R/I_{j}, r \mapsto(r+I_1,...,r+I_n)$ (ist Ringhomomorphismus). Dann gilt:
    \begin{enumerate}[(a)]
        \item $\varphi$ ist surjektiv $\Leftrightarrow$ Die Ideale $I_1,...,I_n$ sind paarweise relativ prim.
        \item $\ker\varphi = \bigcap_{j=1}^{n}I_j$
        \item $\varphi$ ist injektiv $\Leftrightarrow \bigcap_{j=1}^{n}I_j=\{0\}$
    \end{enumerate}
    Insbesondere erhalten wir unter der Vorraussetzung, dass $I_1,..., I_n$ paarweise relativ prim sind, einen Ringidomorphismus
    $$R/\prod_{j=1}^{n}I_j\cong R/I_1\times...\times R/I_n$$
\end{satz}
\begin{proof}
    Das Nullelement in $R/I_j$ ist $I_j$ und das Einselement ist $1+I_j$. Für die bessere Lesbarkeit des Beweises bezeichnen wir diese (unabhängig von $j$) jeweils mit $\overline{0},\overline{1}$.
    \begin{enumerate}[(a)]
        \item "$\Rightarrow$": Sei $\varphi$ surjektiv, seien $i,j\in \{1,...,n\},i\neq j$.
        \newline
         Behauptung: $I_i+I_j=(1)$. Wegen $\varphi$ surjektiv existiert ein $x\in R$ mit $$\varphi(x)=(\overline{0},...,\overline{0},\underbrace{\overline{1}}_{i-\text{te Stelle}},\overline{0},...,\overline{0})\implies x\in I_j.$$
            Außerdem: \begin{align*}
                \varphi(1-x)&=\varphi(1)-\varphi(x)\\
                &=(\overline{1},...,\overline{1})-(\overline{0},...,\overline{0},\underbrace{\overline{1}}_{i-\text{te Stelle}},\overline{0},...,\overline{0}) = (\overline{1},...,\overline{1},\underbrace{\overline{1}}_{i-\text{te Stelle}},\overline{0},...,\overline{0})\\
            &\implies 1-x\in I_i\\
            &\implies 1 = \underbrace{(1-x)}_{\in I_i}+\underbrace{x}_{\in J_i}\in I_i+I_j\implies I_i+I_j=(1)\\
            \end{align*}
        \item "$\Leftarrow$": Seien $I_1,...,I_n$ paarweise relativ prim. 
        \begin{enumerate}
            \item Behauptung: $(\overline{0},...,\overline{0},\underbrace{\overline{1}}_{i-\text{te Stelle}},\overline{0},...,\overline{0})\in \Im \varphi$ für $i=1,...,n$
                Sei $I\in \{1,...,n\}$ fixiert.
                \begin{align*} &\text{ Für } j\neq i \text{ ist } I_i+I_j=(1)\\
                    &\implies \text{ Es existieren } u_j\in I_i, v_j\in V_j \text{ mit } u_j+v_j=1\\
                    &\text{Setze } x:=v_1\cdot...\cdot v_{i-1}\cdot v_{i+1}\cdot...\cdot v_n\\
                    &\implies x\in I_j\text{ für }j\neq i \text{ und } x \\
                    &= (1-u_{1})\cdot...\cdot(1-u_{i-1})(1-u_{i+1}\cdot...\cdot(1-u_n)\\
                    &= 1+z \text{ für ein } z \in I_i\\
                    &\implies \varphi(x)=(\overline{0},...,\overline{0},\underbrace{\overline{1}}_{i-\text{te Stelle}},\overline{0},...,\overline{0})\\
                \end{align*}
                \item \begin{align*}
                    \text{Sei } y=(r_1+I_1,...,r_n+I_n)\\
                    \implies \varphi(r_1+I_1,...,r_n+I_n)&=\varphi(r_1)\varphi(e_1)+...+\varphi(r_n)\varphi(e_n)\\
                    &=(r_1+I_1,\overline{0},...,\overline{0})+...+(\overline{0},...,\overline{0},r_n+I_n)\\
                    &=(r_1+I_1,...,r_n+I_n)=y
                \end{align*}
            \end{enumerate}
            \item $\ker \varphi = \{r\in R|r+I_1=I_1,...,r+I_n\}=I_1\cap ...\cap I_n$
            \item aus (b)
        \end{enumerate}
        Der Rest folgt aus dem Homomorphiesatz.
    \end{proof}
\begin{bsp}
    Seien $R= \Z, I_1=2\Z,I_2=3\Z$. Dann ist 
    $$\varphi:\Z\longrightarrow \Z/2\Z \times \Z/3\Z, a \mapsto (a+2\Z, a+3\Z)$$
    surjektiv wegen $2\Z+3\Z=(1)$ (vgl. Beispiel 1.36). $\ker \varphi=2\Z\cap 3\Z = 6\Z$. D.h. $\varphi$ induziert einen Ringisomorphismus 
    $$\Z/6\Z\cong \Z/2\Z\times \Z/3\Z$$.
\end{bsp}
\section{Teilbarkeit}
\begin{ziel}
    Verallgemeinerung des Konzepts, der Teilbarkeit auf $\Z$ und damit verbundene Bedrifflichkeit (z.B. Primzahl, ggT) auf nullteilerfreie Ringe. Wir zeigen, dass in jedem Hauptidealring ein Analogon des Satzes über die eindeutige Primfaktorzerlegung in $\Z$. 
    \textit{Notation}: In diesem Abschnitt sei $R$ stets ein nullteilerfreier Ring.
\end{ziel}
\begin{definition}
    Seien $a,b\in R$.
    \begin{itemize}
    \item $b$ heißt ein \textbf{Teiler} von $a$ (Notation: $b|a$)$\overset{\text{Def:}}{\Leftrightarrow}$ Es existiert ein $c\in R $ mit $a=bc$.
    \item $a,b$ heißen \textbf{assoziiert} (Notation: $a\widehat= b)\overset{\text{Def:}}{\Leftrightarrow} a|b$ und $b|a$
    \end{itemize}
\end{definition}
\begin{bsp}
    $R=\Z$,$a\in\Z\implies a \widehat= -a$
\end{bsp}
\begin{bem}
    Seien $a,b\in R$. Dann sind äquivalent:
    \begin{enumerate}[(i)]
        \item $a \widehat=b$
        \item Es existiert ein $e\in R^{\times} $ mit $a=be$
        \item $(a)=(b)$
    \end{enumerate}
\end{bem}
\begin{proof}
\begin{itemize}
    \item (i)$\implies$ (ii): Sei $a\widehat=b\implies a|b$ und $b|a\implies $ Es existieren $c,d\in R$ mit $b=ac $ und $a=bd\implies b=ac =bdc\implies b(1-dc)=0$
    \begin{enumerate}
        \item Erster Fall: $b=0 \implies a=0$. Setze $e:=1$. Fertig.
        \item Zweiter Fall: $b\neq 0\underset{R\text{ nullteilerfrei}}{\implies} 1-dc=0\implies cd=1\implies c,d\in R^{\times}$. Setze $e:=d$, dann $a=be$ mit $e\in R^{\times}$
    \end{enumerate}
    \item (ii)$\implies$ (iii): Sei $a =be$ mit $e\in R^{\times} \implies a\in(b) \implies (a)\subseteq (b). $ Wegen $e\in R^{\times} $ ist $b =e^{-1}a\implies b\in (a) \implies (b)\subseteq (a)$
    \item (iii)$\implies$ (i): Sei $(a)=(b) \implies a\in (b) \implies $ Es existiert $c\in R$ mit $a=bc\implies b|a$. Analog: $a|b$. Also: $a\widehat=b$.
\end{itemize}
\end{proof}
\begin{definition}
    Seien $a_1,...,a_n\in R$.
    $d\in R$ heißt ein \textbf{gtößter gemeinsamer Teiler} von $a_1,...,a_n \overset{\text{Def:}}{\Leftrightarrow}$ Die folgenden Bedingungen sind erfüllt: 
    \begin{enumerate}[(GGT1)]
        \item $d|a_1,...,d|a_n $
        \item $c|a_1,...,c|a_n \implies c|d$
    \end{enumerate}
    Wir bezeichnen die Menge der größten gemeinsamen Teiler von $a_1,...,a_n$ mit GGT($a_1,...,a_n$)
\end{definition}
\begin{anm}
    \begin{itemize}
        \item Sind $d_1,d_2\in $ GGT($a_1,...,a_n$), dann folgt$d_1|d_2$ und $d_2|d_1$, also $d_1\widehat=d_2$
        \item Ist $d\in$ GGT($a_1,...,a_n$) und $d'\widehat=d$, dann ist $d'\in $ GGT($a_1,.l..,a_n$)
        \item Ohne zusätzliche Vorraussetzung an $R$ kann man im Allgemeinen nicht erwarten, dass GGT($a_1,...,a_n$)$\neq \emptyset$ 
        (z.B. in $R=\Z[\sqrt{-1}]= \{a+b\sqrt{-3}|a,b\in\Z\}\subseteq \C$ isr GGT($4,2(1+\sqrt{-3})$)$=\emptyset$)
    \end{itemize}
        
\end{anm}
\begin{bem}
    Sei $R$ ein HIR und $a_1,...,a_n\in R$. Dann gilt: 
    \begin{enumerate}[(a)]
        \item GGT($a_1,...,a_n$)$\neq \emptyset$
        \item $d\in $GGT($a_1,...,a_n$)$\Leftrightarrow(d)=(a_1,...,a_n)$
    \end{enumerate}
\end{bem}
\begin{proof}
    \begin{enumerate}[(a)]
        \item $R$ HIR $\implies$ Es existiert $\tilde d\in R$ mit $(a_1,...,a_n)=(\tilde d)$
        Behauptung: $\tilde d \in $GGT($a_1,...,a_n$).
        denn: \begin{enumerate}[(GGT1)]
            \item $a_i \in (a_1,...,a_n)=(\tilde d) \implies \tilde d |a_i$ für $i=1,...,n$
            \item Sei $c\in R$ mit $c|a_1,...,c|a_n$. Wegen $\tilde d \in (a_1,...,a_n)$ existieren $r_1,...,r_n\in R $ mit $\tilde d = r_1a_1+...+r_na_n$. Somit folgt $c|(r_1a_1+...+r_na_n)$, d.h. $c|\tilde d$.
        \end{enumerate}
        \item "$\Rightarrow$": Sei $d\in$ GGT/$a_1,...,a_n$)$\overset{\text{Anm. 2.4}}{\implies} d\widehat=\tilde d \overset{2.3}{\implies}(d)=(\tilde d)=(a_1,...,a_n)$
        \item "$\Leftarrow$": Sei $(d)=(a_1,...,a_n)\implies d\in $ GGT($a_1,..,a_n$) mit selben Argument wie im Beweis von (a).
    \end{enumerate}
\end{proof}
\begin{anm}
    \begin{itemize}
    \item Im Fall $R=\Z$, $a_1,...,a-n\in\Z$ ist GGT($a_1,..,a_n$)$\cap \N_{0}=\{d\}$ für ein $d\in \N_{0}.$ (beachte: $\Z^{\times}=\{-1,1\}.)$ Man nennt dann $d$ den größten gemeinsamen Teiler von $a_1,...,a_n$: $$d=:\operatorname{ggt}(a_1,...,a_n)$$
    \item Im Fall $F=K[t]$ (wobei $K$ Körper, in §3: dies ist ein HIR), $f_1,...,f_n\in K[t]$, nicht alle $f_i=0$, dann existiert ein eindeutig bestimmtes Polynom $d\in K[t]$ mit $d\in$ GGT($f_1,...,f_n$)(beachte: ($K[t]^{\times}=K^{\times})$. Man nennt $$d=: \operatorname{ggT}(f_1,.,,f_n)$$ den größten gemeinsamen Teiler von $f_1,...,f_n$.(Und man setzt ggT$/0,...,0):=0$.)
    \end{itemize}
\end{anm}
\begin{fg}
    Sei $R$ ein HIR, $a,b\in R, d\in$ GGT$(a,b)$. Dann existieren $u,v\in R$ mit $d=ua+vb$
\end{fg}
\begin{proof}
    aus 2.5: $d\in(d)=(a,b)$
\end{proof}
\begin{definition}
    Sei $p\in R\backslash(R^{\times}\cup \{0\})$
    \begin{align*}
        \text{p heißt \textbf{irreduzibel}}\overset{\text{Def:}}{\Leftrightarrow}& \text{ Aus } p=ab \text{ mit } a,b\in R \text{ folgt stets }a\in R^{\times} \text{ oder } b\in R^{\times}\\
        \text{p heißt \textbf{Primelement}}\overset{\text{Def:}}{\Leftrightarrow}& \text{ Aus } p|ab \text{ mit } a,b\in R \text{ folgt stets } p|a \text{ oder } p|b\\
        \Leftrightarrow& (p) \text{ ist ein Primideal}
    \end{align*}
\end{definition}
\begin{anm}
    $p$ irreduzibel bzw. Primelement, $p'\widehat= p\implies p'$ irreduzibel bzw. Primelement.
\end{anm}
\begin{bsp}
    \begin{align*}
        \text{ irreduzible Elemente in }\Z &= \text{ Primzahlen aus }\N \text{ sowie deren Negative}\\
        &= \text{ Primelemente in }\Z
    \end{align*}
\end{bsp}
\begin{frage}
    Zusammenhang zwischen irreduziblen Elemente und Primelementen in $R$?
\end{frage}
\begin{bem} 
    Sei $p\in R\backslash(R^{\times}\cup \{0\})$ ein Primelement. Dann ist $p$ irreduzibel.
\end{bem}
\begin{proof}
    Sei $p=ab$ mit $a,b\in R\implies p|ab \underset{p \text{ Prmideal}}{\implies} p|a \text{ oder } p|b$. Gelte ohne Einschränkung: $p|a$. Außerdem: $a|p$, somit $p\widehat= a$. Nach 2.3 existiert ein $w\in R^{\times }$ mit $a=ep\implies p=ab=epb\implies p(1-eb)=0\underset{p\neq 0}{\overset{R \text{ nullteilerfrei}}{\implies}} 1-eb=0 \implies eb=1$, d.h. $b\in R^{\times}$
\end{proof}
\begin{anm}
    Es gibt Beispiele für irreduzible Elemente, die kein Primelemt sind (vgl. Übungen)
\end{anm}
\begin{bem}
    Sei $R$ ein HIR, $p\in R\backslash(R^{\times}\cup \{0\})$. Dann sind äquivalent:
    \begin{enumerate}[(i)]
        \item $p$ ist irreduzibel
        \item $p$ ist Primelement
    \end{enumerate}
\end{bem}
\begin{proof}
    \begin{itemize}
        \item (ii) $\implies$ (i) aus 2.9
        \item (i) $\implies$ (ii) Sei $p$ irreduzibel. 
        \begin{enumerate}
            \item $(p)$ ist maximales Ideal in $R$, denn: Sei $I\subseteq R $ Ideal mit $(p)\not\subseteq I$. Wegen $R$ HIR existiert $a\in R$ mit $I=(a) \underset{p\in I}{\implies}$ Es existiert $c\in R$ mit $p=ac\implies a\in R^{\times}$ oder $c\in R^{\times}$. Falls $c\in R^{\times}$, dann $8p)=(a)=I$ nach 2.3. Also $a\in R^{\times}$, d.h. $I=(a)=R$ \textit{Widerspruch}
        \item Wegen 1. und 1.24 ist $(p)$ Primideal, d.h. $p$ ist Primelement. 
        \end{enumerate}
    \end{itemize}
\end{proof}
\begin{anm}
    Beweis hat gezeigt: In HIR gilt für $p\in R\backslash(R^{\times}\cup \{0\}): p$ irreduzibel $\Leftrightarrow (p)$ maximales Ideal.
\end{anm}
\begin{frage}
    Wann gilt in $R$ ein Analogon des Satzes über die eindeutige Primfaktorzerlegung in $\Z$?
\end{frage}
\begin{definition}
    $R$ heißt \textbf{faktoriell} $\overset{\text{Def:}}{\implies}$ Jedes $a\in R\backslash(R^{\times}\cup \{0\})$ lässt sich eindeutig bis auf Reihenfolge und Assoziierbarkeit als Produkt von irreduziblen Elementen aus $R$ schreiben, d.h es existieren irreduzible Elemente $p_1,...,p_r\in R$ mit $$a=p_1\cdot...\cdot p_r$$
    und sind $q_1,...,q_s$ irreduzible Elemente mit $a=q_1\cdot ... \cdot q_s$, so ist $r=s$ und nach Umnummerieren ist $p_i\widehat=q_i$ für $i=1,...,r$
\end{definition}
\begin{ziel}
    Jeder HIR ist faktoriell.
\end{ziel}
\begin{definition}
    R heißt \textbf{noethersch} $\overset{\text{Def:}}{\Leftrightarrow} $ Für jede aufsteigende Kette $I_1\subseteq I_2\subseteq ... $ von Idealen in $R$ existiert ein $n\in \N$ mit $I_k=I_n$ für alle $k\geq n$.
\end{definition}
\begin{bem}
    Sei $R$ ein HIR. Dann ist $R$ noethersch.
\end{bem}
\begin{proof}
    Sei $I_1\subseteq I_2 \subseteq ...$ eine aufsteigende Kette von Idealen aus $R$. Setze $I:=\bigcup\limits_{k\geq 1}I_k$
    \begin{enumerate}
        \item $I$ ist ein Ideal in $R$, denn:
        \begin{enumerate}[(J1)]
            \item $0\in I_k $ für alle $k\geq 1\implies 0\in I$
            \item Seien $a,b\in I \implies $ Es existieren $k,l\in\N$ mit $a\in I_k,b\in I_l$. Mit $m:=\max\{k,l\}$ ist $a,b\in I_m\implies a+b\in I_m\subseteq I$
            \item Seien $a\in I,r\in R\implies $ Es existiert ein $k\in\N$ mit $a\in I_k \implies ra\in I_k \subseteq I$
        \end{enumerate}  
            \item Wegen 1. und $R$ HIR existiert ein $a\in R $ mit $i=(a),$ insbesondere $a\in I\implies $ Es existiert ein $N\in\N$ mit $a\in I_n\implies (a)\subset I_n\subset I =(a)\implies I_n = I \implies I_k=I_n$ für alle $k\geq n$.
        \end{enumerate}
    \end{proof}
    \begin{satz}
        Sei $R$ ein HIR. Dann ist $R$ faktoriell.
    \end{satz}
\begin{proof} 
    \begin{enumerate}
        \item Existenz von Zerlegung in irreduzible Elemente. Setze $M:=\{(a)|a\in R\backslash(R^{\times}\cup\{0\}) \text{ besitzt keine Faktorisierung in irreduziele Elemente }\}.$
        \begin{itemize}
            \item Annahme: $M\neq \emptyset $. Es existiert ein bezüglich $\subseteq$ maximales ELement $j\in M$, denn: Andernfalls existiert zu jedem $I\in M$ ein $I'\in M$ mit $I\not\subseteq I'$, das liefert eine unendlich strikt aufsteigende Kette von Idealen in $R$. \textit{Widerspruch} zu $R$ noethersch.
                Es existiert ein $a\in R $ mit $J=(a)$. $a$ ist nicht irreduzibel, denn für $a$ irreduzibel wäre $a$ selbst eine Faktorisierung in irreduzible Elemente $\implies J=(a)\not\in M $\textit{WidersprucH}$\implies $ Es existieren $a_1,a_2\in  R\backslash(R^{\times}\cup \{0\})$ mit $a=a_1a_2 \implies (a)\subseteq (a_1), (a)\subseteq (a_2).$ Wäre $(a)=(a_1)$, dann existiert ein $b\in R^{\times}$ mit $a=a_1b=a_1a_2\implies a_1(a_2-b)=0\overset{R \text{ nullteilerfrei}}{\implies} a_2=b\in R^{\times}$\textit{Widerspruch}. Also: $(a)\not\subseteq (a_1)$, analog $(a)\not\subseteq (a_2) \implies (a_1),(a_2)\not\in M \implies a_1,a_2$ haben Faktorisierung in irreduzible Elemente, also auch $a=a_1a_2$\textit{Widerspruch}. Also: $M\neq \emptyset \implies $ Existenz. 
        \end{itemize} 
        \item Eindeutigkeit der Zerlegung: Sei $a=p_1\cdot ... \cdot p_r=q_1\cdot ...\cdot p_r =q_1\cdot ... \cdot q_s$ mit $p_1,...,p_r,p_1,...,p_s$ irreduzibel. Beweis per Induktion nach $r$:
            \begin{itemize}
                \item Induktionsanfang: $r=0\implies a=1\implies s=0$(sonst$q_1,...,q_s\in R^{\times}$\textit{Widerspruch})
                \item Induktionsannahme: Die Behauptung sei für $0,...,r-1$ bewiesen. 
                \item Induktionsschritt: $p_1|p_1\cdot ... \cdot p_r = q_1\cdot ... \cdot q_s\overset{p_1 \text{Primelememt}}{\implies} $ Es existiert ein $j\in\{1,...,s\}$ mit $p_1|q_j$. Nach Umnummerieren sei $j=1$, also $p_1|q_1$, etwa $q_1=cp_1$ mit $c\in R$. Da $q_1$ irreduzibel ist, folgt $c\in R^{\times}$, also $p_1\widehat=q_1 \implies p_1\cdot ... \cdot p_r = cp_1q_2\cdot ... \cdot q_s \implies p_1(p_2\cdot ... \cdot p_r-cq_2\cdot ... \cdot q_s)=0\underset{R \text{ nullteilerfrei}}{\implies} p_2\cdot ... \cdot p_r0(cq_2)q_3\cdot...\cdot q_s.$ Wegen $c\in R^{\times}$ ist $cq_2$ irreduzibel $\overset{IV}{\implies } r-1=s-1(\implies r=s)$ und nach Umnummerieren ist $p_2\widehat=cq_2\widehat=q_2,p_3\widehat=q_3,...,p_r\widehat=q_r$
            \end{itemize}
        \end{enumerate}
	\end{proof}
	\section{Euklidische Ringe}
\textit{Notation:} In diesem Abschnitt sei $R$ stets ein Ring.
\begin{definition}
    $R$ heißt \textbf{euklidischer Ring} $\overset{\text{Def:}}{\Leftrightarrow} $ $R$ ist nullteilerfrei und es existiert eine Abbildung $\delta:R\backslash \{0\} \longrightarrow \N_{0}$ , so dass gilt: Für alle $f,g\in R,g\neq 0$ existieren $q,r\in R $ mit $f=qg+r$ und $(\delta(r)<\delta(g)$ oder $r=0$). $\delta$ heißt eine \textbf{Normabbildung} auf $R$.

\end{definition}
\begin{bsp} 
    \begin{enumerate}
        \item $R=\Z$ mit $\delta =|\cdot| $ ist ein euklidischer Ring (Bem. 1.5)
        \item $K$ Körper $\implies R=K[t]$ mit $\delta =$deg ist ein euklidischer Ring 
        \item $K $ Körper mit $\delta : K\backslash \longrightarrow \N_0, x\mapsto 1$ ist ein euklidischer Ring ( hier ist $f=fg^{-1}g+0$, hier ist $r=0$)
        \item $R=\Z[I]=\{a+bi|a,b\in \Z\}\subseteq \C$ ist ein euklidischer Ring mit $\delta(x+iy)=x^2+y^2$ ( Ring mit ganzen Gaußschen Zahlen) (vgl. Übungen)
    \end{enumerate}
\end{bsp}
\begin{satz}
    Sei $R$ ein euklidischer Ring. Dann ist $R$ ein Hauptidealring.

\end{satz}
\begin{proof}
Sei $I\subseteq R $ ein Ideal, $I\neq 0$. Es ist $\emptyset \neq \{\delta(a)|a\in I\backslash \{0\}\}\subseteq N_0$. Wähle $a\in I\backslash \{0\}$, so dass $\delta(a) $ minimal. Behauptung: $I=(a)$, denn: \begin{itemize}
    \item "$\supseteq$": Wegen $a\in I $ ist $(a)\subseteq I$
    \item "$\subseteq$": Sei $f\in I\implies $ Es existiert $a,r\in R$ mit $f=qa+r$ und $(\delta(r)<\delta(a)$ oder $r=0$)$\implies r=\underset{\in I}{f}-\underset{\in I}{qa}\in I$. Wegen $\delta(a)$ minimal folgt $r=0\implies f=qa\in (a)$
\end{itemize}
\end{proof}
\begin{anm}
    Es gibt Hauptidealringe, die nicht euklidisch sind (siehe Beispieldatenbank)
\end{anm}
\begin{fg}
     Sei $R$ ein euklidischer Ring. Dann ist $R$ faktoriell.
\end{fg}
\begin{proof}
    $R$ euklidisch $\overset{3.3}{\implies} R $ Hauptidealring $\overset{2.14}{\implies} R $ faktoriell.
\end{proof}
\begin{fg}
    Sei $K$ ein Körper, $f\in K[t], f\neq 0$. Dann besitzt $r$ eine bis auf Reihenfolge der Faktoren eindeutige Darstellung:
    $$ f=c\overset{e_1}{p_1}\cdot ... \cdot \overset{e_r}{p_r}$$
    mit $c\in K^{\times}, r\geq 0, e_1,...,e_r\in \N$ und paarweise verschiedenen normierten irreduziblen Polynomen $p_1,...,p_r$.

\end{fg}
\begin{proof}
    nach 3.2 ist $K[t]$ euklidisch, nach 3.4 also faktoriell.
\end{proof}
\begin{satz}[Euklidischer Algorithmus] Sei $R$ ein euklidischer Ring mit Normabbildung $\delta, a,b \in R\backslash \{0\}$. Wir betrachten eine Folge $a_0,a_1,...$ von Elementen aus $R$, dei induktiv wie folgt gegeben ist: 
    \begin{align*}
        a_0&:= a\\
        a_1&:=b\\
        a_0&:= q_0a_1+a_2 \text{ mit } \delta(a_2)<\delta(a_1) \text{ oder } a_2=0\\
        \text{ Falls }a_2\neq 0: a_1&= q_1a_2+a_3 \text{ mit } \delta(a_3)<\delta(a_2) \text{ oder } a_3=0\\
        &\vdots \\
        \text{ Falls }a_i\neq 0: a_{i-1}&= q_{i-1}a_i+a_{i+1} \text{ mit } \delta(a_{i+1})<\delta(a_i) \text{ oder } a_{i+1}=0\\
        &\vdots\\
    \end{align*}
    Dann existiert ein eindeutig bestimmter Index $n\in \N $ mit $an \neq 0,a_{n+1}=0$. Es ist dann 
    $$d:=a_n\in\operatorname{GGT}(a,b)$$
    Durch Rückwärtseinsetzen lässt sich $d$ als Linearkombinaton von $a,b$ darstellen: 
    $$d=a_n=a_{n-2}-q_{m-2}a_{n-1}=...=ua+vb\text{ mit } u,v\in R$$
    (erweiterter euklidischer Algorithmus)
\end{satz}
\begin{bsp}
    $R=\Z, a=24, b=15$
    \begin{align*}
        24&=1\cdot 15+9\\
        15&=1\cdot 9+6\\
        9&=1\cdot 6 +3\\
        6&=2\cdot 3 +0\\
    \end{align*}
$$\implies \operatorname{ggT}(24,15)=3$$
    Es ist $$3=9-1\cdot 6=9-(15-1\cdot 9)=2\cdot 9 - 1\cdot 15 = 2\cdot(24-1\cdot 15)-15=2\cdot 24-3\cdot 15.$$
\end{bsp}
\begin{proof}[von 3.6]
    Falls $a_i\neq 0$ für alle $i \in \N$, dann wäre $\delta(a_1)>\delta(a_2)> ... $ eine streng monoton fallende unendliche Folge in $\N_0$. \textit{Widerspruch}.
    $\implies$ Es existiert ein eindeutig bestimmtes $n\in\N$ mit $a_n\neq 0,a_{n+1}=0$. 
    Wir betrachten die Gleichungen: 
    \begin{align*}
        (G_0)&\text{ }a_0= q_0a_1+a_2\\
        &\vdots\\
        (G_{n-2})&\text{ } a_{n-2}=q_{n-2}a_{n-1}+a_n\\
        (G_{n-1})&\text{ } a_{n-1}=q_{n-1}a_n
    \end{align*}
    Dann gilt: $a_n|a_{n-1}\overset{(a_{n-2})}{\implies} a_n|(q_{n-2}a_{n-1}+a_n)=a_{m-2}\implies ... \implies a_n|a_1 $ und $ a_n|a_0$.
    Sei $c\in R$ mit $c|a_0$ und $c|a_1\overset{(a_0)}{\implies} c|(a_0-q_0a_1)=a_2\implies ... \implies c|a_n$. 
    Also: $a_n\in $ GGT($a_0,a_1)=$GGT($a,b$). Es ist 
    \begin{align*}
        a_n&=a_{n-2}-q_{n-2}a_{n-1}\overset{G_{n-3})}{=} a_{n-2}-q_{n-2}(a_{n-3}-q_{n-3}a_{n-2})\\
        &=(1+q_{n-2}q_{n-3})a_{n-2}-q_{n-2}a_{n-3}=...=ua+vb
    \end{align*}
    (mit geeigneten $u,v\in R$)
\end{proof}
\begin{satz}[Gauß-Diagonalisierung von Matrizen]
    Sei $R$ ein euklidischer Ring, $A\in M_{n,n}(R)$. 
    Dann gilt: $A$ lässt sich durch wiederholtes Anwenden von elementaren Zeilen- und Spaltenoperationen vom Typ
    \begin{itemize}
        \item Addition des $\lambda$-Fachen einer Zeile/Spalte zu einer anderen Zeile bzw. Spalte 
        \item Zeilen-/Spaltenvertauschung
    \end{itemize}
in eine Matrix der Gestalt 
$$\left(\begin{array}{cccc|c}
    c_1& & & & \\
    & c_2& & &  \\  
    & & \ddots& & \\
    & & & c_r& \\
    \hline & & & & \\
    & & 0& & \\
\end{array}\right)$$
mit $c_1,...,c_r\in R\backslash \{0\}, c_1|c_2|...|c_r$ überführen.
\end{satz}
\begin{proof}
Falls $A=0$, dann fertig. Im Folgenden sei $\underset{=(a_{ij})}{A}\neq 0$. Dei $\delta$ eine Normabbildung auf $R$.
\begin{enumerate}
\item Durch Zeilen- und Spaltenvertauschen erreichen wir $a_{11}\neq 0$ und $\delta(a_{11})\leq \delta(a_{ij})$ für alle $i,j$ mit $a_{ij}\neq 0$.
\item Ziel: Bringe $A$ auf die Form 
$$\left(\begin{array}{c|cccc}
    \ast& &0 & & \\
    \hline& & & &  \\  
    0& & \ast & \\
    & & & & \\
\end{array}\right)$$, wobei links oben Element $\neq 0$ mit minimalen $\delta$ 
\begin{itemize}
    \item \underline{1. Fall:} In der ersten Spalte/Zeile stehen keine Elemente $\neq 0$ außer $a_{11}; $ dann fertig
    \item \underline{2. Fall:} In der ersten Spalte/Zeile stehen noch Elemente $\neq 0$, ohne Einschränkung $a_{21}\neq 0\implies $ Es existiert ein $q\in R$ mit $a_{21}=qa_{11} $ oder $\delta(a_{21}-qa_{11})<\delta(a_{11}). $ Addiere das $(-q$)-fache der 1. Zeile zur 2. Zeile $(\ast\ast)$. $\implies $ Erhalte Matrix $A'=(a_{ij}')$ mit $a_{21}'\neq 0$ oder $\delta(a_{21}')<\delta(a_{11})$. Erhalte durch Zeilen/Spaltenvertauschen eine Matrix 
        $$A''=(a_{ij}'') \text{ mit } a_{11}''\neq 0, \delta(a_{11}'')\leq \delta(a_{ij}'')\text{ für alle } i,j  \text{ mit } a_{ij}''\neq 0$$
        und  $\delta(a_{11}'')=\leq \delta(a_{11})("="$ nur, wenn obige Division aufgefangen und $\delta(a_{11}$ nach $(\ast\ast)$) immer noch minimal). Iteriere dies, dieser Prozess bricht nach endlich vielen Iterationen ab.
        Erhalte eine Matrix der Form 
        $$D= \left(\begin{array}{c|cccc}
            d_{11}& & & & \\
            \hline& & & &  \\  
            0& & \ast& & \\
            & & & & \\
        \end{array}\right)$$ 
        mit $d_{11}\neq 0,\delta(d_{11})\leq d(d_{ij})$ falls $ d_{ij}\neq 0, \delta(d_{11})\leq \delta(a_{11})$
    \end{itemize}
    \item Erreiche $d_{11}|d_{ij}$ für alle $i,j$.
    \begin{itemize}
        \item \underline{1. Fall:} Es gilt bereits $d_{11}|d_{ij}$ für alle $i,j$ dann fertig
        \item \underline{2. Fall:} Es existeiren $i,j$ mit $d_{11}$ nicht Teiler von $d_{ij}\implies$ Es existiert ein $q\in R$ mit $d_{ij}-qd_{11}\neq 0$ und $\delta(d_{ij}-qd_{11})<\delta(d_{11})$ Addiere erste Zeile von $D$ zur $i-ten$ Zeile von $D$, erhalte 
        $$\left(\begin{array}{c|ccccc}
            d_{11}&0&... & 0& ...&0\\
            \hline 0 & & & \ast& & \\  
            \vdots& & & & &\\
            d_{11}& d_{i2}&... & d_{ij}&...&d_{in}\\
            0& & & & & \\
            \vdots& & & \ast& & \\
            0 & & & & &\\
        \end{array}\right)$$. Subtrahiere das $q-$fache der ersten Spalte von der $j-$ten Spalte dieser Matrix, erhalte
        $$D'=(d_{ij}')=\left(\begin{array}{c|ccccccc}
            d_{11}&0&... & 0&-qd_{11}&0& ...&0\\
            \hline 0 & & & & \ast& & & \\  
            \vdots& & & & & & &\\
            0& & & & & & &\\
            d_{11}& \ast& & & d_{ij}-qd_{11}& & &\ast\\
            0& &&& & & & \\
            \vdots& & & & & & & \\
            0 & & & &\ast & & &\\
        \end{array}\right)$$ mit $d_{ij}'=d_{ij}-qd_{11},\delta(d_{ij}')<\delta(d_{11})\leq\delta(a_{11})$. Widerhole die gesamte bisherige Prozedur für die Matrix $D'$. Dieser Prozess bricht nach endlich vielen Schritten ab. Wir erhalten eine Matrix
        $$C= \left(\begin{array}{c|cccc}
            c_{11}& & D& & \\
            \hline& & & &  \\  
            0& & C'& & \\
            & & & & \\
        \end{array}\right)$$ mit $c_{11}\neq 0,\delta(c_{11})\leq \delta(a_{11})$ und $c_{11}|c_{ij}$ für alle $i,j$.
    \end{itemize}
    \item Wende das Verfahren auf $C'$ an (und iteriere dies). Operationen an $C'$ erhalten die Teilbarkeit durch $c_{11}$, d.h. wir können die Matrix auf die Gestalt 
    $$\left(\begin{array}{c|cccc}
        \ast& &0 & & \\
        \hline& & & &  \\  
        0& & \ast & \\
        & & & & \\
    \end{array}\right)$$ mit $c_1|c_2|c_3|...|c_r$ bringen. 
\end{enumerate}
\end{proof}
\begin{bsp}
    \begin{enumerate}[(a)]
        \item $R=\Z$ mit $\delta=|\cdot|$. 
        \begin{align*}
            A&= \begin{gmatrix}[p]
                4 & 3\\
                6 & 5
            \end{gmatrix}
            \leadsto 
            \begin{gmatrix}[p]
                3 & 4\\
                5 & 6
            \end{gmatrix}
            \leadsto
            \begin{gmatrix}[p]
                3 & 1\\
                5 & 1
            \end{gmatrix}\\
            &\leadsto
            \begin{gmatrix}[p]
                1 & 3\\
                1 & 5
            \end{gmatrix}
            \leadsto
            \begin{gmatrix}[p]
                1 & 0\\
                1 & 2
                \rowops 
                \mult{1}{\text{II}-\text{I}}
            \end{gmatrix}
            \leadsto
            \begin{gmatrix}[p]
                1 & 0\\
                0 & 2
            \end{gmatrix}
        \end{align*}
        \item $R=\Q[t] $ mit $\delta = $deg.
            $$A=\begin{gmatrix}[p]
                t-1 & 0\\
                -1 & t-1 
                \rowops
                \swap{0}{1}
            \end{gmatrix}\leadsto
            \begin{gmatrix}[p]
                -1 & t-1\\
                t-1 & 0
                \rowops
                \mult{1}{\text{II}+(t-1)\text{I}}
            \end{gmatrix}\leadsto
            \begin{gmatrix}[p]
                -1 & t-1\\
                0 & (t-1)^2
            \end{gmatrix}\leadsto
            \leadsto
            \begin{gmatrix}[p]
                -1 & 0\\
                0 & (t-1)^2 
            \end{gmatrix}$$
    \end{enumerate}
\end{bsp}
\textbf{Erinnerung an LA1} 
\begin{itemize}
    \item Zeilen-/bzw. Spaltenoperationene wie in 3.8 lassen sich durch Multiplikation mit Elementarmatrizen
        $$E_{ij}=\begin{pmatrix}
            1 & & \\
            &\ddots&\\
            \lambda & &1\\
        \end{pmatrix}$$,
        $$P_{ij}=\begin{gmatrix}[p]
            1 & & & & & & & & & &\\
            & \ddots& & & & & & & & &\\
            & & 1 & & & & & & & &\\
            & & & 0 & & & & 1 & & &\\
            & & & & 1 & & & & & &\\
            & & & & &\ddots& & & & &\\
            & & & & & & 1 & & & &\\
            & & & 1 & & & & 0 & & & \\
            & & & & & & & & 1 & &\\
            & & & & & & & & & \ddots& \\
            & & & & & & & & & &1\\
        \end{gmatrix}$$
    von links bzw. rechts beschreiben.
    \item Determinanten lassen sich auch von quadratischen Matrizen mit Einträgen in $R$ bilden (via Leibnizformel). Es ist $A\tilde A=\tilde A = \det (A)E_n$, wobei $\tilde A$ adjungte Matrix zu $A$. Insbesondere: $A\in M_{n,n}(R)$ invertierbar (d.h. es existiert $B\in M_{n,n}(R)$ mit $AB=BA=E_{n})\Leftrightarrow \det(A)\in R^{\times} $(vgl. LA1 Def. 4.63)
    \end{itemize}
\begin{definition}
    \begin{align*}
    \operatorname{GL}(R)&=\{A\in M_{n,n}(R)|A \text{ ist invertierbar }\}
    &= \{A\in M_{n,n}(R)|\text{det}(A)\in R^{\times}\}
    \end{align*}
    ist eine Gruppe bzgl. Multiplikation, die \textbf{allgemeine lineare Gruppe} über $R$ von Rang $n$.
\end{definition}
\begin{definition}
    $A$ heißt \textbf{äquivalent} z $B$ ($A\sim B)\overset{\text{Def:}}{\Leftrightarrow} $ Es existieren $S\in $GL$_{n}(R),T\in$ GL$_{n}(R)$ mit $B=SAT^{-1}.$ Falls $m=n$, so heißt $A$ \textbf{ähnlich} zu $B$ $(A\approx B)\overset{\text{Def:}}{\Leftrightarrow}$ Es existiert $S\in$ GL$_{n}(R)$ mit $B=SAS^{-1}$
\end{definition}
\begin{anm}
    \begin{itemize}
        \item $\sim, \approx$ sind Äquivalenzrelationen auf $M_{m,n}(K),$ nzw. $M_{n,m}(K)$
        \item $K$ Körper, $A,B\in M_{n,n}(K), \mathrm{C}$ Basis von $K^n,\mathrm{D}$ Basis von $K^m,f:K^n\longrightarrow K^m $ lineate Abbildung mit $M_{\mathrm{D}}^{\mathrm{C}}(f)=A$. Dann: $A\sim B\Leftrightarrow $ Es existieren Basen $\mathrm{C'},\mathrm{D'}$ von $K^n$ bzw. $K^m$ mit $M_{\mathrm{D'}}^{\mathrm{C'}}(f)=B$ (d.h. $A,B$ beschreiben bzgl. geeignter Basen dieselber lineare Abbildun)
    \end{itemize}
\end{anm}
\begin{frage}
    Gibt es innerhalb einer Äquivalenzklasse bzgl. $\sim$ einen besonders schönen Vertreter?
\end{frage}
\begin{fg}
    Sei $R$ ein euklidischer Ring, $A\in M_{m,n}(R).$ Dann existieren $c_1,...,c_r\in R\backslash\{0\}$ mit $c_1|c_2|...|c_r$ und 
    $$A\sim \left(\begin{array}{ccc|c}
        c_1& &0&0\\
        & \ddots & &\\
        0& &c_r& \\
        \hline & & 0& 0
    \end{array}\right)$$
\end{fg}
\begin{proof}
    Umformungen in 3.8 korrespondieren zur Multiplikation mit Elementarmatrizen von links bzw. rechts mit Determinante $\in\{-1,1\}$ (diese sind also invertierbar)
\end{proof}
\begin{anm}
    Um durch Zeilen- bzw. Spaltenoperationen zu 
    $$A\sim \left(\begin{array}{ccc|c}
        c_1& &0&0\\
        & \ddots & &\\
        0& &c_r& \\
        \hline & & 0& 0
    \end{array}\right)$$
    zu gelangen, darf man auch Zeilen bzw. Spalten mit $\lambda \in R^{\times } $ multiplizieren 
    $$\begin{gmatrix}[p]
        1& & & & & &\\
        & \ddots  & & & & &\\
        & & 1 & & & &\\
        & & & \lambda & & & \\
        & & & &1 & & \\
        & & & & &\ddots &\\
        &&&&&&1
    \end{gmatrix}$$
    (invertierbar für $\lambda \in R^{\times})$ 
    d.h.: Im Allgemeinen zu 3.8 ist diese Operation jetzt auch erlaubt.
\end{anm}
\begin{erin}
Sei $K$ ein Körper, $A\in M_{n,n}(K)$. Dann gelten:
\end{erin}
\begin{itemize}
    \item Rang $A=r\implies$ $$ A\sim 
    \left(\begin{array}{c|c} E_r& 0\\ 
        \hline 0& 0
    \end{array}\right)$$
    \item $S\in$ GL$_{n}(K),T\in$ GL$_{n}(K)\implies $ Rang($SAT^{-1}$)=Rang($A$)
\end{itemize}
Es folgt für $A,B\in M_{m,n}(K):$
$$A\sim B \Leftrightarrow \text{ Rang } A= \text{ Rang }B$$
\begin{ziel}
    Klassifikation von Matrizen aus $M_{m,n}(R), R $ euklidischer Ring, bis auf Äquivalenz.
\end{ziel}
\begin{definition}
    Sei $A\in M_{m,n}(R), 1\leq k\leq m, 1\leq l\leq n$. 
    \begin{itemize}
        \item $B\in M_{k,l}(R)$ heißt eine \textbf{Untermatrix von A }$\overset{\text{ Def:}}{\Leftrightarrow} B$ entsteht aus $A$ durch Streichen $m-k$ Zeilen und $n-l$ Spalten. 
        \item Ist $B\in M_{l,l}(R)$ eine quadratische Untermatrix von $A$ mit ($l\leq \min\{m,n\}$), dann heißt $\det(B)$ ein \textbf{Minor $l$-ter Stufe }von $A$. 
        \item Fit$_{l}(A)=(\det(B)|B \text{ ist }l\times l-\text{Untermatrix von }A )\subseteq R$ (d.h. das von allen Minoren $l$-ter Stufe von $A$ erzeugte Ideale in $R$) heißt das \textbf{$l$-te Fittingideal} von $A$.
    \end{itemize}
\end{definition}
\begin{bsp}
    $$A=\begin{gmatrix}[p] 1 &2\\
        3&4
    \end{gmatrix}\in M_{2,2}(\Z)$$
    \begin{itemize}
    \item Fit$_{1}(A)=(\det(1),\det(2),\det(3),\det(4))=(1,2,3,4)=(1)=\Z$
    \item $$\operatorname{Fit}_{2}(A)=\left(\det\begin{gmatrix}[p]1 &2 \\ 3& 4\end{gmatrix}\right)=(-2)=2\Z$$
    \end{itemize}
\end{bsp}
\begin{satz}[Fittings Lemma]
    Seien $A\in M_{m,n}(R),S\in \GL_{n}(R), T\in \GL_{n}(R),l\leq \min\{m,n\}$. Dann gilt:
    $$\Fit_{l}(A)=\Fit_{l}(SA)=\Fit_{l}(AT)$$
\end{satz}
\begin{proof}
    \begin{enumerate}
        \item $\Fit_{l}(SA)\subseteq \Fit_{l}(A)$, denn:
        $$A=(a_{ij})\in M_{m,n}(R), S=(s_{ij})\in\GL_{m}(R),SA=(b_{ij})\in M_{m,n}(R).$$ 
        Seien $A\leq i_1<i_2<...<i_l\leq m , 1\leq j_1<j_2<...<j_l\leq n.$ Wir betrachten die $l\times l$-Untermatrix
        $$B=\begin{gmatrix}[p]
            b_{i_1,j_1} & ... & b_{i_1,i_l}\\
            \vdots & &\vdots \\
            b_{i_l,j_1} & ... & b_{i_l,j_l}
        \end{gmatrix}$$ von $SA$
        \begin{align*}
            \implies \det(B) &=\det\begin{gmatrix}[p]
            \sum\limits_{r_1=1}^{m}s_{i_1,r_1}a_{r_1,j_1} & ... & \sum\limits_{r_1=1}^{m}s_{i_1,r_1}a_{r_1,j_l}\\
            b_{i_2,j_1} & ... & b_{i_2,j_l}\\
            \vdots & & \vdots\\
            b_{i_l,j_1}& ... & b_{i_l,j_l}
        \end{gmatrix}\\
        &= \sum_{r_1=1}^{m}s_{i_1,r_1}\cdot \det\begin{gmatrix}[p]
            a_{r_1,j_1}& ... & a_{r_1,j_l}\\
            b_{i_2,j_1} & ... & b_{i_2,i_l}\\
            \vdots & &\vdots \\
            b_{i_l,j_1} & ... & b_{i_l,j_l}
        \end{gmatrix}\\
        &= \sum_{r_l=1}^{m}...\sum_{r_1=1}^{m}s_{i_1,r_1}\cdot ... \cdot s_{i_l,r_l}\underbrace{\det\begin{gmatrix}[p]
            a_{r_1,j_1}& ... & a_{r_1,j_l}\\
            \vdots & &\vdots \\
            a_{i_l,j_1} & ... & a_{i_l,j_l}
        \end{gmatrix}}_{=\left\{\begin{array}{c}
            0, \text{ falls }i\neq j\text{ existieren mit }r_i=r_j\\
            \pm  \text{ein Minor }l\text{-ter Stufe von }A
        \end{array}\right.}\in \Fit_{l}(A)
    \end{align*}
    $\implies \Fit_{l}(SA)\subseteq \Fit_{l}(A).$ 
    \item Wende 1. auf $S^{-1}\in \GL_{m}(R),SA\in M_{m,n}(R) $ an
    $$\implies \Fit_{l}(S^{-1}(SA))\subseteq \Fit_{l}(SA), \text{ also } \Fit_{l}(A)\subseteq \Fit_{l}(SA)$$
    \item $\Fit_{l}(A)=\Fit_{l}(A^t),$ also $\Fit_{l}(AT) = \Fit_{l}((AT)^t)=\Fit_{l}(T^tA^t)\overset{2.}{=} \Fit_{l}(A^t)=\Fit_{l}(A)$
    \end{enumerate}
\end{proof}
\begin{fg}
    Seien $A,B\in M_{m,n}(R)$ mit $A\sim B$. Dann gilt: $\Fit_{l}(A)=\Fit_{l}(B)$ für alle $A\leq l \leq \min\{m,n\}$
\end{fg}
\begin{proof}
    $A\sim B\implies $ Es existieren $S\in\GL_{m}(R), T\in\GL_{n}(R)$ mit $ B=SAT^{-1}\implies \Fit_{l}(B)=\Fit_{l}(SAT^{-1})\underset{3.15}{=}\Fit_{l}(AT^{-1})\underset{3.15}{=}\Fit_{l}(A)$
\end{proof}
\begin{bem}
    Sei $R$ ein nullteilerfreier Ring, 
    $$A=\left(\begin{array}{cccc|c}
        c_1& & & &0\\
        & c_2& & & \vdots \\  
        & & \ddots& & \\
        & & & c_r&0 \\
        \hline & & & & \\
        0&\hdots &\ &0 & 0\\
    \end{array}\right)\in M_{m,n}(R)$$, mit mit $c_1,...,c_r\in R\backslash \{0\}, c_1|c_2|...|c_r$.
    Dann gilt :
    $$\Fit_{l}(A)=\left\{\begin{array}{cc}
        (c_1\cdot ... \cdot c_l) & \text{,falls } 1\leq l\leq r\\
    (0) & \text{, falls} r<l\leq \min\{m,n\}\end{array}\right.$$
    Insbesondere gilt: $\Fit_{r}(A)\subseteq \Fit_{r-1}(A)\subseteq ... \subseteq \Fit_{1}(A)$
\end{bem}
\begin{proof}
    \begin{itemize}
    \item Für $l>r$ erhält jede $l\times l$-Untermmatrix von $A$ stets eine Nullzeile, d.h. $\Fit_{l}(A)=(0)$
    \item $l\leq r$: Die einzigen $l\times l$-Untermatrizen von $A$, die keine Nullzeile/-spalte enthalten, sind von der Form
        $$\begin{gmatrix}[p]
            c_{i_1}& &0\\
            & \ddots &\\
            0& &c_{i_l}
        \end{gmatrix}$$
        mit $1\leq i_1<i_2<...<i_l\leq r$
        \begin{align*}
            &\implies \Fit_{l}(A)=(c_{i_1}\cdot ... \cdot c_{i_l})| 1\leq i_1<i_2...<i_l\leq r)\\
            &\implies (c_1\cdot ...\cdot c_l)\subseteq \Fit_{l}(A)\\
            \intertext{ Umgekehrt folgt wegen $ 1\leq i_1<i_2<...<i_l\leq r: i_1\geq 1,i_2\geq 2, ..., i_l\geq l$}
            &\implies c_1|c_{i_1},...,c_l|c_{i_l}\implies c_1\cdot ... \cdot c_l|c_{i_1}\cdot ... \cdot c_{i_l} \implies (c_{i_1}\cdot ...\cdot c_{i_l})\subseteq (c_1\cdot ...\cdot c_l)\\
            &\implies \Fit_{l}(A) \subseteq (c_1\cdot ... \cdot c_l)
        \end{align*}
    \end{itemize}
\end{proof}
\begin{satz}[Elementarteilersatz über euklidischen Ringen]
    Sei $R$ ein euklidischer Ring, $A\in M_{m,n}(R)$. Dann existieren $c_1,...,c_r\in R\backslash\{0\}$ mit $c_1|c_2|...|c_r$, so dass 
    $$A\sim \left(\begin{array}{cccc|c}
        c_1& & & &0\\
        & c_2& & & \vdots \\  
        & & \ddots& & \\
        & & & c_r&0 \\
        \hline & & & & \\
        0&\hdots &\ &0 & 0\\
    \end{array}\right)$$
$r$ ist eindeutig bestimmt, $c_1,...,c_r$ sind eindeutig bestimmt bis auf Assoziietheit. $c_1,...,c_r$ heißen \textbf{Elementarteiler von A}
\end{satz}
\begin{proof}
    \begin{enumerate}
        \item Existenz aus 3.12
        \item Eindeutigkeit von $r$:
            Sei 
            $$A\sim \left(\begin{array}{cccc|c}
                c_1& & & &0\\
                & c_2& & & \vdots \\  
                & & \ddots& & \\
                & & & c_r&0 \\
                \hline & & & & \\
                0&\hdots &\ &0 & 0\\
            \end{array}\right)$$ 
            und 
            $$A\sim \left(\begin{array}{cccc|c}
                d_1& & & &0\\
                & d_2& & & \vdots \\  
                & & \ddots& & \\
                & & & d_s&0 \\
                \hline & & & & \\
                0&\hdots &\ &0 & 0\\
            \end{array}\right)$$
            mit $c_1,...,c_r,d_1,...,d_s\in R\backslash\{0\}$ mit $c_1|c_2|...|c_r,d_1|d_2,...,|d_s$,
        $$\overset{3.16}{\underset{3.17}{\implies}} \Fit_{l}(A)=\left\{\begin{array}{cc} (c1\cdot ...\cdot c_l) & l\leq r\\ (0) & \text{ sonst }\end{array}\right. =\left\{\begin{array}{cc} (d_1\cdot ...\cdot d_l)\\ (0) & \text{ sonst }\end{array}\right.$$ für alle $l\in\{1,...,\min\{m,n\}\}$
        $$ \implies r=\max\{l\in\{1,...,\min\{m,n\}\}|\Fit_{l}(A)\neq (0)\}=s$$
        \item $c_l\widehat=d_l$ für $l=1,...,r$ per Induktion nach $l$:
        \begin{itemize}
             \item IA: $\Fit_{1}(A)= (c_1)=(d_1)\overset{2.3}{\implies} c_1\widehat=d_1$
             \item IS: $\Fit_{l}(A)=(c_1\cdot...\cdot c_l)=(d_1\cdot ... \cdot d_l)\implies c_1\cdot ...\cdot c_l\widehat=d_1\cdot ... \cdot d_l,$ außerdem ist nach IV.
                $$c_1\widehat=d_1,...,c_{l-1}\widehat=d_{l-1}\implies c_1\cdot ... \cdot c_l=\underbrace{d_1\cdot...\cdot d_{l-1}}_{c_1\cdot ... \cdot c_{l-1}f \text{ für ein }f\in R^{\times}}d_l\cdot e\text{ für ein }e\in R^{\times}$$
                $$\implies \underbrace{c_1\cdot ... \cdot c_{l-1}}_{\neq 0}(c_l-d_lef)=0\implies c_l=d_lef \implies c_l\widehat= d_l$$
        \end{itemize}
        \end{enumerate}
    \end{proof}
\begin{satz}
    Sei $R$ ein euklidischer Ring, $A,B\in M_{m,n}(R)$. Dann sind äquivalent:
    \begin{enumerate}[(i)]
        \item $A\sim B$
        \item Die Elementarteiler von $A$ und $B$ stimmen bis auf Assoziietheit überein
        \item $\Fit_{l}(A)=\Fit_{l}(B)$ für alle $ 1\leq l \leq \min\{m,n\}$
    \end{enumerate}
\end{satz}
\begin{proof}
    \begin{itemize}
        \item (i)$\implies$ (iii): aus 3.16
        \item (iii)$\implies$ (ii): Seien $c_1,...,c_r,d_1,...,d_s$ die Elementarteiler von $A$ bzw. $B$. Insbesondere
        $$A\sim \left(\begin{array}{cccc|c}
            c_1& & & &0\\
            & c_2& & & \vdots \\  
            & & \ddots& & \\
            & & & c_r&0 \\
            \hline & & & & \\
            0&\hdots &\ &0 & 0\\
        \end{array}\right)$$
        und
        $$b\sim \left(\begin{array}{cccc|c}
            d_1& & & &0\\
            & d_2& & & \vdots \\  
            & & \ddots& & \\
            & & & d_r&0 \\
            \hline & & & & \\
            0&\hdots &\ &0 & 0\\
        \end{array}\right)$$
        Argumentiere nun wie im Beweis von 3.18 in 2. und 3.
        \item (ii) $\implies$ (i) Sei
        $$A\sim \left(\begin{array}{cccc|c}
            c_1& & & &0\\
            & c_2& & & \vdots \\  
            & & \ddots& & \\
            & & & c_r&0 \\
            \hline & & & & \\
            0&\hdots &\ &0 & 0\\
        \end{array}\right)$$
        und
        $$B\sim \left(\begin{array}{cccc|c}
            d_1& & & &0\\
            & d_2& & & \vdots \\  
            & & \ddots& & \\
            & & & d_r&0 \\
            \hline & & & & \\
            0&\hdots &\ &0 & 0\\
        \end{array}\right)$$
        mit $c_1\widehat=d_1,...,c_r\widehat=d_r$, etwa $d_1=\lambda_1c_1,...,d_r0\lambda_rc_r$ mit $\lambda_1,...,\lambda_r\in R^{\times}$
        \begin{align*}
            \implies \left(\begin{array}{cccc|c}
            d_1& & & &0\\
            & d_2& & & \vdots \\  
            & & \ddots& & \\
            & & & d_r&0 \\
            \hline & & & & \\
            0&\hdots &\ &0 & 0\\
        \end{array}\right)&=\left(\begin{array}{cccc|c}
            \lambda_1c_1& & & &0\\
            & \lambda_2c_2& & & \vdots \\  
            & & \ddots& & \\
            & & & \lambda_rc_r&0 \\
            \hline & & & & \\
            0&\hdots &\ &0 & 0\\
        \end{array}\right)\\
        &=\underbrace{\begin{gmatrix}[p]
            \lambda_1 & & & & &\\
            & \ddots & & & &\\
            & & \lambda_r & & & \\
            & & & 1 & & \\
            & & & & \ddots & \\
            & & & & & 1
        \end{gmatrix}}_{\in \GL(m,R)}\left(\begin{array}{cccc|c}
            c_1& & & &0\\
            & c_2& & & \vdots \\  
            & & \ddots& & \\
            & & &   c_r&0 \\
            \hline & & & & \\
            0&\hdots &\ &0 & 0\\
        \end{array}\right)\\
    \end{align*}
    $$A\sim \left(\begin{array}{cccc|c}
        c_1& & & &0\\
        & c_2& & & \vdots \\  
        & & \ddots& & \\
        & & & c_r&0 \\
        \hline & & & & \\
        0&\hdots &\ &0 & 0\\
    \end{array}\right)\sim \left(\begin{array}{cccc|c}
        d_1& & & &0\\
        & d_2& & & \vdots \\  
        & & \ddots& & \\
        & & & d_r&0 \\
        \hline & & & & \\
        0&\hdots &\ &0 & 0\\
    \end{array}\right)\sim B$$
    \end{itemize}
\end{proof}
\begin{anm}
    Satz 3.19 beinhaltet Insbesondere den Fall, das $R=K$ ein Körper ist. Die Elementarteiler von $A\in M_{m,n}(K)$ sind bis auf Assoziietheit: $\underbrace{1,...,1}_{r \text{ Stück }}0,...,0$ (mit $r=$ Rang $A$). D.h. $A\sim B \Leftrightarrow $ Rang $A= $ Rang $B$ 
\end{anm}
\begin{bsp} $$A=\begin{gmatrix}[p]
    6 & -2 \\
    -2 & 2
\end{gmatrix}, B=\begin{gmatrix}[p]
    4 & 8\\
    4 & 6
\end{gmatrix}\in M_{2,2}(\Z)$$
\begin{align*}
    \Fit_{1}(A)&=(6,-2,-2,2)=(2),\\
    \Fit_{2}(A)&=(\det A)=(8)\\
    \Fit_{1}(B)&=(4,8,4,6)=(2)\\
    \Fit_{2}(B)&=(\det B)=(-8)=(8)
\end{align*}
$\implies A\sim B$
Es ist $(c_1)=\Fit_{1}(A), (c_1,c_2)=\Fit_{2}(A)=(8)=(2)$
D.h.: $c_1=2,c_2=4$ sind Elementarteiler von $A$ (bzw. von $B$), Insbesondere sind $$A,B\sim \begin{gmatrix}[p] 2 & 0 \\ 0& 4\end{gmatrix}$$
\end{bsp}
\part{Normalformen und Endomorphismen}
\begin{frage}
    Sei $K$ ein Körper, $V$ euklidischer $K$-VR und $\varphi\in\End(V).$
    Wie einfach kann man $M_B(\varphi)$ bekommen durch geeignete Wahl einer Basis $B$? In Termen von MAtrizen: Suche möglichst einfache Vertreter der Äquivalenzklasen bezüglich "$\approx$".
\end{frage}
\section{Invarianten-und Determinantenteiler}
\begin{notation}
    In diesem Abschnitt sei $K$ stets ein Körper und $n\in\N$.
\end{notation}
\begin{frage}
    Seien $A,B\in\M_{n,n}(K).$ Wann ist $A\approx B$?
\end{frage}
\begin{definition}
    Sei $A\in\M_{n,n}(K)$. 
    $$P_A:=tE_n-A\in\M_{n,n}(K[t]) \text{ heißt die \textbf{charakteristische Matrix} von }A$$
\end{definition}
\begin{anm}
    Insbesondere ist $\chi_{A}^{\text{char}}=\det(P_A).$ Hierbei bezeichnet $\chi_{A}^{\text{char}}$ das charakteristische Polynom von $A$.
\end{anm}
\begin{satz}[Satz von Frobenius]
    Seien $A,B\in\M_{n,n}(K)$. Dann sind äquivalent:
    \begin{enumerate}[(i)]
        \item $A\approx B$ (in $\M_{n,n}(K)$) 
        \item $P_A\sim P_B$ (in $M_{n,n}(K[t])$)
    \end{enumerate}
\end{satz}
\begin{proof}
    \begin{enumerate}[(i) $\implies$ (ii):]
        \item [(i) $\implies$ (ii):] Sei $A\approx B\implies $ Es existiert ein $S\in\GL_n(K)$ mit $B=SAS^{-1}$ 
        \begin{align*}
            \implies& P_B=tE_n-B=tE_n-SAS^{-1}=StE_nS^{-1}-SAS^{-1}=S\underbrace{tE_n-A}_{P_A}S^{-1}\\
            \implies& P_B\approx P_A\implies P_B \sim P_A
        \end{align*}
        \item[(ii) $\implies$ (i):] Sei $P_A\sim P_B.$
        \begin{enumerate}
           \item  Wir konstruieren $R\in\M_{n,n}(K)$ mit $AR=RB$. Nach Vorraussetzung existieren $S,T\in\GL_{n}(K[t])$ mit $P_A=SP_BT^{-1}$, d.h. $SP_B=P_AT$
            $$ \implies S(tE_n-B)=(tE-n-A)T (\ast)$$
            Wir schreiben $S,T$ in der folgenden Form:
            $$S=\sum\limits_{i=0}^{m}t^{i}S_i, T=\sum\limits_{i=0}^{m}t^{i}T_i \text{ mit } S_i,T_i\in\M_{n,n}(K)$$
            \begin{align*}
                \implies S(tE_n-B)&=\sum\limits_{i=0}^{m}t^{i}S_i(zE_n-B)\\
                &= \sum\limits_{i=0}^{m}(t^{i+1}S_i-t^{i}S_iB)\\
                &= \sum\limits_{i=1}^{m+1}t^{i}S_{i-1}-\sum\limits_{i=0}^{m}t^{i}S_iB\\
                &= \sum\limits_{i=0}^{m+1}(S_{i-1}-S_iB)t^{i} \text{ mit } S_{i-1},S_{m+1}:=0.\\
                (tE_n-B)&= (tE_n-A)\sum\limits_{i=0}^{m}t^{i}T_i\\
                &= \sum{i=0}^{m}(t^{i+1}T_i-t^{i}AT_i)\\
                &= \sum\limits_{i=0}^{m+1}(T_{i-1}-AT_i)t^{i}\\
                \overset{(\ast)}{\implies} \sum\limits_{i=0}^{m+1}(S_{i-1}-S_iB)t^{i}&=\sum\limits_{i=0}^{m+1}(T_{i-1}-AT_i)z^{i}\\
                \implies S_{i-1}-S_iB &= T_{i-1}AT_{i} \text{ für } 0\leq i\leq m+1\\
                \implies A_iS_{i-1}-A^{i}S_iB&=A^{i}T_{i-1}-A^{i+1}T_i \text{ für } 0\leq i\leq m+1\\
                \implies \sum\limits_{i=0}^{m+1}(A^{i}S_{i-1}-A^{i}S_iB)&= \sum\limits_{i=0}^{m+1}(A^{i}T_{i-1}-A^{i+1}T-i)\\
                &= (A^{°}T_{i-1}-AT_0)+(AT_o-A^2T_1)\\
                &+...+ (A^{m+1}T_m-A^{m+2}T_{m+1})\\
                &= A^{°}T_{i-1}-A^{m+2}T_{m+1}=0.\\
                \implies \sum\limits_{i=0}^{m+1}A^{i}S_{i-1}&= \sum\limits_{i=0}^{m+1}A^{i}S_{i}B\\
                \overset{S_{m+1}=0}{\underset{S_{-1}=0}{\implies}} \sum\limits_{i=0}^{m+1}A^{i}S_{i-1} &=\sum\limits_{i=0}^{m}A^{i}S_{i}B\\
                \implies A\left(\sum\limits_{i=0}^{m}A^{i}S_i\right)&=\left(\sum_{i=0}^{m}A^{i}S_i\right)B\\
            \end{align*}
            Setze $R:=\sum\limits_{i=0}^{m}A^{i}S_i$, dann $AR=RB.$
            \item Wir zeigen: $R\in\GL_n(K)$ (wegen $AR=RB$ folgt dann $A=RBR^{-1}$, also $A\approx B$, fertig.)
                Nach Vorraussetzung ist $S\in\GL_{n}(K[t])$. 
                $$\text{ Es existiert } M\in\GL_{n}(K[t]) \text{ mit } SM=E_n, M=\sum\limits_{i=0}^{m}t^{i}M_i \text{ mit } M_i\in\M_{n,n}(K),$$
                ohne Einschränkung $m$ wie vorhin.
                \begin{enumerate}
                    \item [Behauptung:] Mit $N:=\sum\limits_{j=0}^{m}RB^{j}M_j\in\M_{n,n}(K)$ gilt $RN=E_n$, d.h. $N\in\GL_n(K)$
                    \item [denn:] Es ist $RN=\sum\limits_{j=0}^{m}RB^{j}M_j.$ Wegen $RB\overset{1.}{=}AR$ folgt $RB^{j}=RBB^{j-1}=ARB^{j-1}=...=A^{j}R$
                        $$\implies RN=\sum\limits_{j=0}^{m}A_jRM_j=\sum\limits_{j=0}^{m}A^{j}\left(\sum\limits_{i=0}^{m}A^{i}S_i\right)M_{j}=\sum\limits_{i,j=0}^{m}A^{i+j}S_iM_j$$
                        Wegen $SM=E_n$ folgt $\left(\sum\limits_{i=0}^{m}t^{i}S_i\right)\left(\sum\limits_{j=0}^{m}t^{j}M_j\right)$ $=E_n.$\\
                        $S_0M_0+\sum\limits_{k=0}\left(\sum_{i+j=k}S_iM_j\right)t^{k}=E_n$\\
                        $\overset{\text{Koeffizentenvergleich}}{\implies} S_0M_0=E_n, \sum\limits_{i+j=k}S_iM_j=0$ für $K\geq 1.$\\
                        $\implies RN=\sum\limits_{i,j=0}^{m}A^{i+j}S_iM_j=S_0M_0+\sum\limits_{k=1}^{2m}A^k\underbrace{\sum\limits_{i+j=k}S_iM_j}_{=0}=E_n\\\implies$ Behauptung.
                \end{enumerate}
        \end{enumerate}
    \end{enumerate}
    \end{proof}  
\begin{bem}
    Sei $A\in\M_{n,n}(K).$ Dann gilt: 
    \begin{enumerate}[(a)]
        \item Es gibt bestimmte normierte Polynome $c_1(A),...,c_n(A)\in K[t]$ mit 
         $$ P_A\sim \begin{gmatrix}[p]
            c_1(A) & &0\\
            & \ddots & \\
            0 & & c_n(A)
         \end{gmatrix}$$ mit $c_1(A)|c_2(A)|...|c_n(A).$ $c_1(A),...,c_n(A)$ heißen die \textbf{Invariantenteiler} von $A$.
         \item Es gibt eindeutig bestimmte normierte Polynome $d_1(A),...,d_n(A)\in K[t]$ mit 
            $$\Fit_{l}(P_A)=(d_l(A))\text{ für }l=1,...,n$$
            Es ist $d_l(A)=$ggT($\det(B)$| $B$ ist $l\times l$-Untermatrix von $P_A$)
            Insbesondere ist $D_n(A)=\chi_{A}^{\text{char}}.$ $d_1(A),...,d_n(A)$ heißen die \textbf{Determinantenteiler} von $A$.
        \end{enumerate}
    \end{bem}
\begin{proof}
    \begin{enumerate}[(a)]
        \item $K[t]$ ist ein Euklidischer Ring (Bsp. 3.2).\\
            $\overset{\text{Satz 3.18}}{\implies}$ Es existieren $\tilde{c_1},...,\tilde{c_r}\in K[t]\setminus \{0\}$ mit 
            $$P_A\sim \begin{gmatrix}[p]
                \tilde{c_1}& & & & &\\
                & \ddots & & & & \\
                & & \tilde{c_r} & & &\\
                & & & 0 & & \\
                & & & & \ddots & \\
                & & & & & 0
            \end{gmatrix}$$ mit $\tilde{c_1}|...|\tilde{c_r}.$ Es ist $\Fit_n(P_A)=(\det P_A)=(\chi_{A}^{\text{char}})\neq (0) \implies r=n$ und $\Fit:n(P_A)\overset{3.16}{=}(\tilde{c_1}\cdot ... \cdot \tilde{c_n}).$ Wegen $\tilde{c_i}\neq 0$ für $i=1,...,n$ existieren normierte Polynome $c_i(A), i=1,...,n$ mit $c_i(A)\widehat{=} \tilde{c_i}.$
            $$\implies P_A\sim \begin{gmatrix}[p]
                c_1(A) & &0\\
            & \ddots & \\
            0 & & c_n(A)
         \end{gmatrix}.$$
         Eindeutigkeit: $c_1'(A),...,c_n'(A)\in K[t]$ normiert mit $c_1'(A)|c_2'(A)|...|c_n'(A)$ und 
        $$ P_A\sim \begin{gmatrix}[p]
            c_{1}'(A) & & \\
            & \ddots &\\
            & & c_{n}'(A)
        \end{gmatrix}$$
  $\implies c_i'(A)\subseteq c_i(A)$ für $i=1,...,n \underset{c_i(A),c_i'(A) \text{ normiert}}{\implies} c_i'(A)=c_i(A)$ für $i=1,...,n$
  \item $K[t]$ HIR nach Satz 3.3 $\implies \Fit_l(P_A), l=1,...,n$ sind Hauptideale und nach 3.16, 3.17 ist $\Fit_l(P_A)=(c_1(A)\cdot ... \cdot c_l(A))$ für $l=1,...,n$, insbesondere ist $\Fit_l(P_A)\neq 0.$ Erzeuger der Hauptidealringe $\Fit_l(P_A)$ sind eindeutig bis auf Assoziiertheit (2.3) $\implies $ Es existieren eindeutig bestimmte Polynome $d_1(A),...,d_n(A)\in K[t]$ mit $\Fit_l(P_A)=(d_l(A))$ für $l=1,...,n.$
    Es ist \begin{align*}
        \Fit_l(P_A)=&(\det(B)|B \text{ ist }l\times l\text{-Untermatrix von }P_A)\\
        \overset{2.5}{=}& (\text{ggT}(\det(B)|B \text{ ist }l\times l\text{-Untermatrix von }P_A )\\
        =& d_l(A)
    \end{align*}
    $$\overset{d_l\text{ normiert}}{\underset{\text{ggT normiert}}{\implies}}d_l(A)=\text{ggT}(...).$$
        \end{enumerate}
    \end{proof}
\begin{anm}
    Also: \begin{align*}
        \text{Invariantenteiler von }A&=\text{ normierte Elementarteiler von }P_A\\
        \text{Determinantenteiler von }A&= \text{ normierten Erzeuger der Fittingideale von} P_A
    \end{align*}
\end{anm}
\begin{Folgerung} Sei $A\in \M_{n,n}(K).$ 
    \begin{align*}
        \intertext{Dann gilt:}
        d_l(A)&=c_1(A)\cdot...\cdot c_l(A) \text{ für } l=1,..,n\\
        \intertext{Insbesondere gilt }
        \chi_{A}^{\text{char}}&= d_n(A)\cdot ... \cdot c_n(A)\\
        \intertext{sowie}
        d_1(A)&|...|d_n(A),\\
        \Fit_n(P_A)&\subseteq \Fit_{n-1}(P_A)\subseteq ... \subseteq \Fit_1(P_A)
    \end{align*}
\end{Folgerung}
\begin{satz}[Invariantenteilersatz]
    Seien $A,B\in\M_{n,n}(K).$ Dann sind äquivalent:
    \begin{enumerate}[(a)]
        \item $A\approx B$ 
        \item Die Invariantenteiler von $A$ stimmen mit den Invarianten von $B$ überein: 
            $$c_1(A)=c_1(B),...,c_n(A)=c_n(B)$$
        \item Die Determinantenteiler von $A$ stimmen mit den Determinantenteilen von $B$ überein: 
            $$ d_1(A)=d_1(B),...,d_n(A)=d_n(B)$$
    \end{enumerate}
\end{satz}
\begin{proof}
    Folgt aus Satz von Frobenius und Satz 4.3 
\end{proof}
\begin{bsp}
    Sei $$A=\begin{gmatrix}[p] 0& 1& 3\\ 
        3 & 1& -4 \\
        -2 & 1 & 5 
    \end{gmatrix}\in\M_{3,3}(\Q)$$
    Es ist 
    $$P_A=\begin{pmatrix}
        t & -1 & -3 \\
        -3 & t-1 & 4\\
        2 & -1 & t-5
    \end{pmatrix}\in \M_{3,3}(\Q[t])$$
    Bestimmen der Determinantenteiler von $A$:
    \begin{align*} d_1(A)&=\ggT(-1,...,)=1\\
        d_2(A) &= \ggT((-1)\cdot 4 - (-3)(t-1),(-3)(-1)-2(t-1),...)\\
        &= \ggT(\underbrace{3t-7,-2t+5}_{\text{teilerfremd}},...,)=1\\
        d_3(A)&= \chi_{A}^{\text{char}}=...=(t-2)^3\\
        &\implies c_1(A)=1, c_2(A)=1,c_3(A)=(t-2)^3
    \end{align*}
    Sei $$B=\begin{pmatrix}
        1 & 1 & 2 \\
        1 & 1 & -2\\
        -1 & 1 & 4 
    \end{pmatrix}\in \M_{3,3}(\Q)\implies 
    P_B=\begin{pmatrix}
        t-1 & -1 & -2 \\
        -1 & t-1 & 2\\
        1 & -1 & t-4 
    \end{pmatrix}$$
Bestimmen der Invariantenteiler von $B$:
\begin{align*}
    P_B&=\begin{gmatrix}[p]
        t-1 & -1 & -2 \\
        -1 & t-1 & 2\\
        1 & -1 & t-4 
    \rowops 
    \swap{0}{1}\end{gmatrix} \sim 
    \begin{gmatrix}[p]
        -1 & t-1 & 2\\
        t-1 & -1 & -2 \\
        1 & -1 & t-4 
        \rowops 
        \mult{1}{\text{II}+(t-1)\text{II}}
        \mult{2}{\text{III}+\text{I}}
    \end{gmatrix}\\
    &\sim \begin{gmatrix}[p]
        -1 & t-1 & 2\\
        0 & (t-1)^2-1 & 2(t-1)-2 \\
        0 & t-2 & t-2 
    \end{gmatrix}
    \sim \begin{gmatrix}[p]
        -1 & 0 & 0\\
        0 & t^2-2t & 2t-4 \\
        0 & t-2 & t-2 
        \rowops
        \swap{1}{2}
    \end{gmatrix}\\
    &\sim \begin{gmatrix}[p]
    -1 & 0 & 0\\
    0 & t^2 & t-2 \\
    0 & t^2-2t & 2t-4 
\end{gmatrix}
\overset{\text{3. SP-2.SP}}{\sim}
\begin{gmatrix}[p]
    -1 & 0 & 0\\
    0 & t^2 & 0 \\
    0 & t^2-2t & -t^2+4t-4 
\end{gmatrix}\\
&\sim \begin{gmatrix}[p]
    -1 & 0 & 0\\
    0 & t^2 & 0 \\
    0 & 0 & -(t-2)^2
\end{gmatrix}\sim \begin{gmatrix}[p]
    -1 & 0 & 0\\
    0 & t^2 & 0 \\
    0 & 0 & (t-2)^2
\end{gmatrix}
\end{align*}
\begin{align*}
    \implies c_1(B)&=1, c_2(B)=t-2,c_3(B)=(t-2)^2\\
    d_1(B)&=1, d_2(B)=c_1(B)c_2(B)=t-2,\\
    d_3(B)&=c_1(B)c_2(B)c_3(B)=(t-2)^3-\chi_{\text{char}}^{B}
\end{align*}
Also $A\not\approx B.$
\end{bsp}
\begin{bem}
    Seien $A,B\in\M_{n,n}(K), K$ Teilkörper eines Körpers $L$. Dann sind äquivalent:
    \begin{enumerate}[(i)]
        \item $A\approx B$ in $\M_{n,n}(K)$
        \item $A\approx B$ in $\M_{n,n}(L)$
    \end{enumerate}
\end{bem}
\begin{proof}
    Übung.
\end{proof}
\section{Normalformen}
\begin{notation}
    In diesem Abschnitt sei $K$ stets ein Körper.
\end{notation}
\begin{ziel}
    Suche möglichst einfache Matrizen, die vorgegebene Invarianten- bzw. Determinantenteiler haben.
\end{ziel}
\begin{definition}
    $g=t^n+a_{n-1}t^{n-1}+...+ a_1t+a_0\in K[t],n\geq 1.$
    $$B_g:=\begin{gmatrix}[p]
        0 & & & & &-a_0\\
        1 & 0 & & & &-a_1\\
        & 1 & \ddots & & & \vdots\\
        & & & \ddots & & \vdots\\
        & & & &0 & -a_{n-2}\\
        & & & &1& -a_{n-1}
    \end{gmatrix}\in \M_{n,n}(K),$$ (für $=1: B_g=(-a_0)$) heißt die \textbf{Begleitmatrix} zu $g$.
\end{definition}
\begin{bem}
    Sei $g\in K[t]$ nicht konstant, normiert und $\deg(g)=n.$ Dann ist $c_1(B_g)=...=a_{n-1}(b_g)=1, c_n(B_g)=g,$ also 
    $$ P_{B_g}\sim \begin{gmatrix}[p]
        1 & & & \\
        & \ddots & & \\
        &  & 1 & \\
        & & & g
    \end{gmatrix}$$ 
    und $d_1(B_g)=...=d_{m-1}(B_g)=1, d_n(B_g)=\chi_{B_g}^{\text{char}}=g.$
\end{bem}
\begin{proof}
    Sei $g=t^n+a_{n-1}t^{n-1}+...+a_0$.
    \begin{enumerate}[1.]
        \item $d_{n-1}(B_g)=1$, denn: 
        $$\begin{gmatrix}[p]
            t & & & & & a_0\\
            -1 & t & & & &a_1\\
            & -1 & \ddots & & & \vdots\\
            & & & \ddots & & \vdots\\
            & & & &t & a_{n-2}\\
            & & & &-1& t+a_{n-1}
        \end{gmatrix}.$$
        Streiche erste Zeile, letze Spalte von $P_{B_g}$, erhalte $(n-1)\times(n-1)$-Untermatrix
        $$C= \begin{gmatrix}[p]
            -1 & t & & & \\
            & -1 & t & & \\
            & & \ddots &\ddots& \\
            & & & & t\\
            & & & &-1
        \end{gmatrix}$$
        mit $\det(C)=(-1)^{n-1}\implies d_{n-1}(B_g)=1$
        \item Wegen $d_1(B_g)|d_2(B_g)|...|d_{n-1}(B_g)=1$ nach 4.4 folgt $d_1(B_g)=...=d_{n-1}(B_g)=1,$ außerdem: $1=d_{n-1}(B_g)=c_1(B_g)\cdot...\cdot c_{n-1}(B_g)$ nach 4.4, d.h. $c_1(B_g)=...=c_{n-1}(B_g)=1.$
        \item Es ist $d_n(B_g)=\chi_{B_g}^{\text{char}}.$ Wir zeigen per Induktion nach $n$, dass $\chi_{B_g}^{\text{char}}=g.$
        \begin{enumerate}[\textbf{IA:}]
            \item [\textbf{IA:} $n=1$:] $g=t+a_0, B_g(-a_0)\implies \chi_{B_g}^{\text{char}}=t+a_0=g$
            \item [\textbf{IS:}] 
            \begin{align*}
                \chi_{B_g}^{\text{char}}=&\det\begin{gmatrix}[p] t & & & & & a_0\\
                -1 & t & & & &a_1\\
                & -1 & \ddots & & & \vdots\\
               & & & \ddots & & \vdots\\
               & & & &t & a_{n-2}\\
                & & & &-1& t+a_{n-1}
           \end{gmatrix}\\
            =& \underbrace{t\cdot \det\begin{gmatrix}[p]
                t &  & & & &a_1\\
               -1&  & \ddots & & & \vdots\\
                & & & \ddots & & \vdots\\
                & & & &t & a_{n-2}\\
                & & & &-1& t+a_{n-1}
            \end{gmatrix}}_{P_{B_{\tilde{g}}}}\\
            +&(-1)^{n+1}a_0\underbrace{\det\begin{gmatrix}[p]
                -1 & t & & & \\
            & -1 & t & & \\
            & & \ddots &\ddots& \\
            & & & & t\\
            & & & &-1
        \end{gmatrix}}_{=(-1)^{n-1}}\\
        \intertext{ wobei $\tilde{g}:=t^{n-1}+a_{n-1}t^{n-2}+...+a_2t+a_1$}
        \overset{\text{\textbf{IV}}}{=}& t(t^{n-1}+a_{n-1}t^{n-2}+...+a_2t+a_1)a_0+a_0\\
        =& t^n+a_{n-1}t^{n-1}+...+a_1t+a_0\\
        =& g.
    \end{align*}
\end{enumerate}
    \item Wegen $d_n(B_g) \overset{4.4}{=}c_1(B_g)\cdot...\cdots c_n(B_g)$ und $c_1(B_g)=...=c_{n-1}(B_g)$ folgt $c_n(B_g)=d_n(B_g)=g.$
\end{enumerate}
\end{proof}
\begin{bem}
    Seien $g_1,...,g_r\in K[t]$ normiert, nichtkonstant mit $g_1|g_2|...|g_r, n:=\deg(g_1)+...+\deg(g_r)$
    $$B_{g_1,...,g_r}:= \begin{gmatrix}[p]
        B_{g_1}& & &\\
        & B_{g_2}& & \\
        & & \ddots & \\
        & & & B_{g_r}
    \end{gmatrix}\in \M_{n,n}(K).$$
    Dann gilt: $c_1(B_{g_1,...,g_n})=1,...,c_{n-r}(B_{g_1,...,g_r})=1, c_{n-r+1}(B_{g_1,...,g_r})=g_1,...,c_n(B_{g_1,...,g_r})=g_r.$
\end{bem}
\begin{proof}
\begin{align*}
    P_{B_{g_1,...,g_r}}&=\begin{gmatrix}[p]
        P_{B_{g_1}}& & &\\
        & P_{B_{g_2}}& & \\
        & & \ddots & \\
        & & & P_{B_{g_r}}
    \end{gmatrix}\\
    &\sim \begin{gmatrix}[p]
        1 & & & & & & & & & & \\
        & \ddots &  & & & & & & & \\
        & & 1  & & & & & & &\\
        & & & & g_1& & & & & \\
        & & & & & \ddots & & & & \\
        & & & & & &1 & & & & \\
        & & & & & & &\ddots & & \\
        & & & & & & & &1 & \\
        & & & & & & & & &g_r
    \end{gmatrix}\\
    &\sim \begin{gmatrix}[p]
        1 & & & & & \\
        & \ddots &&&&\\
        & & 1 & & & \\
        & & & g_1 & &\\
        & & & & \ddots & \\
        & & & & & g_r
    \end{gmatrix}\\
    &\implies \text{Behauptung}
\end{align*}
\end{proof}
\begin{satz}[Frobenius Normalform]
Sei $A\in \M_{n,n}(K).$ Dann existiert ein eindeutig bestimmtes $r\in \N_0$, sowie eindeutig bestimmte nicht konstante Polynome $g_1,...,g_r\in K[t]$, mit $g_1|g_2|...|g_r$ und $A\approx B_{g_1,...,g_r}$. $g_1,...,g_r$ sind genau die nichtkonstanten Invariantenteiler von $A$. $B_{g_1,...,g_r}$ heißt die \textbf{Frobenius-Normalform (FNF)} von $A$.
\end{satz}
\begin{proof}
    \begin{enumerate}[1.]
        \item Existenz: \begin{align*} \text{Setze }k&:= \max\{l\in\{1,...n\}|c_l(A)=1\}\\
            r&:= n-k\\
            g&:= c_{k+i}(A) \text{für }i=1,...,r
        \end{align*}
        \begin{align*}
            &\implies n = \deg(\chi_{A}^{\text{char}})=\deg(d_n(A))=\deg(c_1(A)\cdot...\cdot c_n(A))=\deg(g_1\cdot ... \cdot g_r)=\deg(g_1)+...+\deg(g_r)\\
            &\implies B_{g_1,...,g_r} \text{ ist }n\times n\text{-Untermatrix mit Invariantenteilern }1,...,1,g_1m...,g_r\text{(nach 5.3), d.h. mit denselben Inbvariantenteilern wie }A\\
            &\overset{\text{Inv.teiler}}{\underset{\text{Satz}}{\implies}} A\approx B_{g_1,...,g_n}
        \end{align*}
        \item Eindeutigkeit: $A\approx B_{g_1,...,g_r}\approx B_{h_1,...,h_s},$ wobei $h_1,...,h_s$ nichtkonstant, normiert mit $h_1|...|h_s$ $ \overset{\text{5.3}}{\underset{\text{Inv.teilersatz}}{\implies}}r=s,h_i=g_i$ für $i=1,...,r.$
    \end{enumerate}
\end{proof} 
\begin{bsp} (vgl. Bsp 4.6)
    \begin{enumerate}[(a)]
        \item $$A=\begin{pmatrix}
            0 & 1& 3\\
            3 & 1 & -4 \\
            -2 & 1 & 5
        \end{pmatrix}\in \M_{3,3}(\Q)$$
        \begin{align*}
            &\implies c_1(A)=1,c_2(A)=1,c_3(A)=(t-2)^3=t^3-6t^2+12t-8=:g_1\\
            &\implies A\approx B_{g_1}=\begin{pmatrix} 0& 0& 8\\
                1 & 0 & -12\\
                0 & 1 & 6\\
            \end{pmatrix} (\text{FNF von }A)
        \end{align*}
        \item $$A=\begin{pmatrix}
            1 & 1& 2\\
            1 & 1 & -2 \\
            -1 & 1 & 4
        \end{pmatrix}\in \M_{3,3}(\Q)$$
        \begin{align*}
            &\implies c_1(A)=1,c_2(A)=t-2=:g_1,c_3(A)=(t-2)^2=t^2-4t+4=:g_2\\
            &\implies A\approx B_{g_1,g_2}=\left(\begin{array}{c|cc}
                2 & 0& 0\\
                \hline 0 & 0& -4\\
                0 & 1 & 4
            \end{array}\right) (\text{FNF von }A)
        \end{align*}
        \item $$A=\begin{pmatrix}
            4 & -1 & -2 & 3 \\
            -1 & 5 & 2 & -4\\
            0 & 1 & 3 &-1\\
            1 & 2 & 2 & 1 
        \end{pmatrix} \in M_{4,4}(\Q)$$
$\implies c_1(A)=1,c_2(A)=1,c_3(A)=t-3=:g_1,c_4(A)=(t-3)^2(t-2)=t^3-8t^2+21t-18=:g_2$
$$A\approx \left(\begin{array}{c|ccc}
    3 &0 &0 &0\\
    \hline 0 & 0 & 0 & 18\\
    0 & 1 & 0 & -21\\
    0 & 0 & 1 & 8
\end{array}\right)\text{ (FNF von }A)$$
\end{enumerate}
\end{bsp}
\begin{frage}
    $K[t]$ ist faktorieller Ring. Invariantenteiler können in Primfaktoren zerlegt werden. Nutzen für Normalform? $\leftrightsquigarrow$ Weierstrass-Normalform.
\end{frage}
\begin{bem}
    Sei $g\in K[t], g= h_1\cdot...\cdot h_k$ mit $h_1,...,h_k\in K[t]$ normiert, nicht paarweise teilerfremd
    $$ \implies B_g\approx \begin{pmatrix}
        B_{h_1}& & \\
        & \ddots & \\
        & & B_{h_k}
    \end{pmatrix}$$
\end{bem}
\begin{proof} Für $k=1$ ist die Aussage trivial, im Folgenden sei $k\geq 2.$
    \begin{enumerate}[1.]
        \item Sei $C:=$ rechte Seite, dann ist 
        \begin{align*}
            P_c&=\begin{pmatrix}
                P_{B_{h_1}}& & \\
                & \ddots & \\
                & & P_{B_{h_k}}
            \end{pmatrix}\\
            &\sim \begin{gmatrix}[p]
                1 & & & & & & & & & & \\
                & \ddots &  & & & & & & & \\
                & & 1  & & & & & & &\\
                & & & & h_1& & & & & \\
                & & & & & \ddots & & & & \\
                & & & & & &1 & & & & \\
                & & & & & & &\ddots & & \\
                & & & & & & & &1 & \\
                & & & & & & & & &h_k
            \end{gmatrix}\\
            &\sim \begin{gmatrix}[p]
            1 & & & & & \\
            & \ddots & & & &\\
            & & 1 & & & \\
            & & & h_1 & &\\
            & & & & \ddots & \\
            & & & & & h_k
        \end{gmatrix}=:H\\
        P_{B_g}&\sim \begin{gmatrix}[p]
            1 & & & \\
            & \ddots &  &\\
            & & 1&\\
            & & &g
        \end{gmatrix}=: G
    \end{align*}
    \item $G,H$ haben dieselben Fittingideale, \newline denn: Sei $n=\deg(g),$ insbesondere $G,H\in\M_{n,n}(K[t])$\\
    \begin{itemize}
        \item $\Fit_{n}(H)=(\det(H))=(h_1,...,h_k)=(g)=(\det G)=\Fit_n(G)$
        \item $\Fit_{1}(G)= \Fit_{n-1}(G)=(1) (\text{ nach }3.17)$
        \item $\Fit_{n-1}(H)\supseteq (h_1\cdot...\cdot h_{i-1}h_{i+1}\cdot... \cdot h_k| i=1,...,k)=$ \newline$(\underbrace{\ggT(h_1\cdot...\cdot h_{i-1}h_{i+1}\cdot ...\cdot h_k|i=1,...,k)}_{=: f})$
    \end{itemize}
    \begin{enumerate}[\textbf{Behauptung:}]
        \item [\textbf{Behauptung:}] $f=1$
        \item [\textbf{Annahme:}] $f\neq 1 \implies f$ nichtkonstant, d.h. es existiert ein Primelement $p\in K[t]$ mit $p|f.$ Sei $i\in \{1,...,k\}\implies p|h_1\cdot...\cdot h_k\implies p|h_j$ für ein $j\neq i.$ Außerdem $p|h_!\cdot...\cdot h_{j-1}h_{j+1}\cdot...\cdot_{k}\implies p|h_l$ für ein $l\neq j\implies \ggT(h_j,h_l)\neq 1$ \textit{Widerspruch}. Wegen $\Fit_{n-1}(H)\subseteq \Fit_{n-2}(H)\subseteq ... \subseteq \Fit_{1}(H)$ ( aus 3.16 und 3.17, d.h. $\Fit_{1}(H)=...=\Fit_{1}(H)=(1)).$
        \item Wegen 2. ist $G\sim H \implies P_{B_g}\sim P_C\implies B_g\approx C.$ 
    \end{enumerate}
\end{enumerate}
\end{proof}
\begin{satz}[Weierstrass-Normalform]
    Sei $A\in\M_{n,n}(K).$ Dann existiert ein eindeutig bestimmtes $m\in\N$, Polynome $h_1,...,h_m\in K[t]$, die Potenzen von irreduziblen Polynomen sind, sodass 
    $$A\approx B_{h_1,...,h_m}.$$
    $h_1,...,h_m$ sind eindeutig bis auf die Reihenfolge bestimmt und heißen die \textbf{Weierstrassteiler} von $A$.
    $B_{h_1,...,h_m}$ heißt eine \textbf{Weierstrass-Normalform} von $A$ (WNF). $H_1,...,H_M$ sind die Potenzen irreduzibler Polynome, die in den Primfaktorzerlegungen der nichtkonstanten Invariantenteiler von $A$ auftauchen.
\end{satz}
\begin{proof}
    \begin{enumerate}[1.]
        \item Existenz: (Algorithmus zur Herstellung der WNF)\newline 
        Seien $g_1,...,g_r\in K[t]$ die nichtkonstanten Invariantenteiler von $A$ mit $g_1|g_2|...|g_r.$
        $$A\approx B_{g_1,...,g_r}=\begin{pmatrix}  B_{g_1}& & &\\
            & B_{g_2}& & \\
            & & \ddots & \\
            & & & B_{g_r}
        \end{pmatrix}$$
        Nach 3.5 (Primfaktorzerlegung in $K[t]$) existieren für $i=1,...,r$ teilerfremde Polynome $h_{i,1},...,h_{i,ki}$ die Potenzen irreduzibler Polynome sind, sodass $g_i=h_{i,1}\cdot...\cdot h_{i,ki}$
        $$\overset{5.7}{\implies} A\approx \begin{pmatrix}
            B_{h_1,1}& & & & & & \\
            & \ddots & & & & & \\
            & & B_{h_1,k_1}& & & &\\
            & & &\ddots & & &\\
            & & & &B_{h_r,1}& &\\
            & & & & &\ddots & \\
            & & & & & &B_{h_r,k_r}
        \end{pmatrix}$$
        \item Eindeutigkeit von $m$ sowie von $h_1,...,h_m$ auf Reihenfolge: 
        $$\text{Sei }A\approx  \begin{pmatrix} B_{h_1}& & &\\
        & B_{h_2}& & \\
        & & \ddots & \\
        & & & B_{h_m}
    \end{pmatrix}, \text{ wobei }h_1,...,h_m \text{ Potenzen irreduzibler Polynome}$$
    Wie sortieren $h_1,...,h_m$ so, dass $h_1=p_1^{e_1},...,h_k=p_k^{e_k}, p_1,...,p_k$ irreduzibel, normiert, paarweise verschieden, so dass alle weiteren Polynome $h_k+1,...,h_m$ Potenzen von $p_1,...,p_k$ sind mit kleinerem oder gleichem Exponenten. Setze $f_1:=h_1\cdot ...\cdot h_k(=\operatorname{kgV}(h_1,...,h_m))$
    $$\implies A\underset{5.7}{\approx}
    \begin{pmatrix} 
        B_{f_1}& & & \\
        & B_{h_{k+1}}& & \\
        & & \ddots &\\
        & & & B_{h_m}
    \end{pmatrix}, $$ $$f_1h_{k+1}\cdot...\cdot h_m=h_1\cdot ... \cdot h_m, f_1 \text{ normiert von Grad }\geq 1$$
    Wende dieses Verfahren auf die Matrix 
    $$\begin{pmatrix}
         B_{h_{k+1}}& & \\
         & \ddots &\\
        & & B_{h_m}
    \end{pmatrix}$$ an. Nach Umsortieren von $h_{k+1},...,h_{m}$ wie oben erhalten wir $f_2\in K[t]$ mit 
    $$A\approx \begin{pmatrix}
        B_{f_1}& & & & \\
        & B_{f_2}& & &\\
        & &B_{h_l}& &\\
        & & &\ddots & \\
        & & & &B_{h_m}
    \end{pmatrix}, f_2|f_1, f_1f_zh_l\cdot...\cdot h_m=h_1\cdot ... \cdot h_m$$
    $f_1$ normiert vom Grad $\geq 1,$ sodass $f_r|f_{r-1}|...|f_1, f_1\cdot f_r=h_1,...,h_m$ und 
    $$A\approx \begin{pmatrix}
        B_{f_1}& &\\
        &\ddots &\\
    & &B_{f_r}\end{pmatrix}
    \approx \begin{pmatrix}
        B_{f_r}& &\\
        &\ddots & \\
    & &B_{f_1}\end{pmatrix}= B_{f_r,...,f_1}$$
    $\overset{\text{Eind.}}{\underset{\text{der FNF}}{\implies}} f_1,...,f_r$ eindeutig bestimmt. Über die Faktorisierung von $f_1,...,f_r$ bekommt man $m$ und $h_1,...,h_m$ (bis auf Reihenfolge) zurück. 
    $$\implies m \text{ eindeutig bestimmt, }h_1,...,h_m \text{ eindeutig, bis auf Reihenenfolge}.$$
    \end{enumerate}
\end{proof}
\begin{bsp}
    \begin{enumerate}[(a)]
        \item $$A=\begin{pmatrix}
         -2 & 1 & 5\\
        1 & 1 &-2\\
        3 & 1 & 6
        \end{pmatrix}\in\M_{3,3}(\Q)$$
        $\implies c_1(A)=1,c_2(A)=1,c_3(A)=(t-1)(t-2)^2.$ Mit $h_1=t-1,h_2=(t-2)^2=t^2-4t+4$ ist 
        $$A\approx B_{h_1,h_2}=
            \left(\begin{array}{c|cc}
                1 & 0 &0\\
                \hline 0 & 0& -4\\
                0 & 1 & 4
            \end{array}\right) \text{ (WNF von } A)$$
            \item (vgl. Bsp. 5.5 (c))
            $$A\approx \begin{pmatrix}
                4 & -1 & -2 & 3 \\
                -1 & 5 & 2 & -4\\
                0 & 1 & 3 &-1\\
                1 & 2 & 2 & 1 
            \end{pmatrix} \in M_{4,4}(\Q)$$
            $\implies c_1(A)=1,c_2(A)=1,c_3(A)=t-3,c_4(A)=(t-3)^2(t-2).$ Mit $h_1:=t-3,h_2:=t-2,h_3:=(t-3)^2=t^2-6t+9$
            $$A\approx B_{h_1,h_2,h_3}=\left(\begin{array}{cccc}
                3 & 0 & 0& 0\\
                 0 &2  & 0 & 0\\
                0 & 2 &  0& 0\\
                0& 0&  0& -9\\
                0 & 0& 1 & 6
            \end{array}\right)\text{ WNF von }A$$
        \end{enumerate}
    \end{bsp}
    \begin{ziel} Einfachere Normalform, falls $\chi_{A}^{\text{char}}$ in Linearfaktoren zerfällt (und damit alle Weierstrassteiler Potenzen linearer Polynome sind)
    \end{ziel}
    \begin{bem}
        Seien $\lambda \in K, f=(t-\lambda)^e\in K[t].$ Dann gilt:
            $$B_f\approx \begin{pmatrix}
                \lambda &  & &0\\
                1 &\ddots & &\\
                & \ddots & & \\
                0 & & 1 & \lambda 
            \end{pmatrix}=: J(\lambda, e )\in \M_{e,e}(K) (e=1: J(\lambda, 1)=(\lambda))$$
            Eine Matrix der Form $J(\lambda, e)$ heißt eine \textbf{Jordanmatrix} über $K$.
        \end{bem}
    \begin{proof}
    Setze $J:=J(\lambda, e).$\newline
    Behauptung: $B_f,J$ haben dieselben Determinantenteiler,\newline
    denn: Es ist $$P_J=\begin{pmatrix}
        t-1 & & & \\
        -1 &\ddots & & \\
        & \ddots & &\\
        & & -1 & t-1 
    \end{pmatrix}\implies d_e(J)=(t-\lambda)^{e}=d_e(B_f)$$
    Es ist $$\det \begin{pmatrix}
        t-1 & & & \\
        -1 &\ddots & & \\
        & \ddots & &\\
        & & -1 & t-1 
    \end{pmatrix}= (-1)^{e-1}\implies d_{e-1}(J)=1$$
    Wegen 4.4 ist $$d_1(J)=...=d_{e-2}(J)=1\overset{\text{Invariantenteiler}}{\underset{\text{satz}}{\implies} }B_f\approx J$$
    \end{proof}
    \begin{satz}[Jordansche Normalform]
        Sei $A\in \M_{n,n}(K),\chi_{A}^{\text{char}}$ zerfalle in $K[t]$ in Linearfaktoren. Dann existieren Jordanmatrizen $J-1=J(\lambda_1,e_1),...,J_m(\lambda_m,e_m)$ über $K$, sodass
        $$ A\approx \begin{pmatrix}
            J_1& & & \\
            & J_2 & &\\
            & & \ddots & \\
            & & & J_m
        \end{pmatrix}=: J.$$
    Hierbei sind $\lambda_1,...,\lambda_m$ die (nicht notwendig paarweise verschiedenen) Eigenwerte von $A$($=$ Nullstellen von $\chi_{A}^{\text{char}}$). $J_1,...,J_m$ sind bis auf Reihenfolge eindeutig bestimmt. Die Matrix $J$ heißt eine \textbf{Jordansche Normalform (JNF)} von $A$.
    \end{satz}
    \begin{proof}
        \begin{enumerate}[1.]
            \item Existenz: \begin{align*}
                \intertext{Es ist $\chi_{A}^{\text{char}}=d_n(A)=c_1(A)\cdot ...\cdot c_n(A)$}
                &\overset{\text{Vor}}{\implies} c_1(A),...,c_n(A) \text{ zerfallen alle in Linearfaktoren}\\
                &\implies \text{Alle Weierstrassteiler }h_1,...,h_m \text{ von } A \text{ sind Potenzen linearer Polynome: }h:=(t-\lambda_i)^{e_i}\\
                &\text{ für ein }\lambda_i\in K,e_i\in \N\\
                \intertext{ Wegen $h_1\cdot...\cdot h_m=c_1(A)\cdot ...\cdot c_n(A)=\chi_{A}^{\text{char}}$ sind $\lambda_i$ genau die Nullstellen von $\chi_{A}^{\text{char}}$ und damit genau die Eigenwerte von $A$. Setze $J_i:=J(\lambda_i,e_i)\overset{5.10}{\implies} B_{h_i}\approx J_i$ ( für $i=1,...,m$)}
                A&\approx \begin{pmatrix}
                    B_{h_1}& & \\
                    & \ddots & \\
                    & & B_{n_m}
                \end{pmatrix} \approx \begin{pmatrix}
                    J_1 & &\\
                    & \ddots & \\
                    & & J_m
                \end{pmatrix}
            \end{align*}
    \item Eindeutigkeit von $J_1,...,J_m$ bis auf Reihenfolge: folgt aus Eindeutigkeit der WNF bis auf Reihenfolge von $h_1,...,h_m$
    \end{enumerate}
    \end{proof}
    \begin{anm}
        \begin{itemize}
        \item Üblicherweise gruppiert man in der JNF Jordanmatrizen zu gleichen EW zusammen (zu einem Block mit aufsteigenden $e_i$)
        \item Es gilt : $A$ diagonalisierbar $\Leftrightarrow$ JNF von $A$ ist eine Diagonalmatrix (denn: "$\Leftarrow$ " trivial, " $ \Rightarrow$ " da Diagonalmatrizen bereits in JNF sind (mit $1\times 1$-Jordanmatrizen))
        \end{itemize}
    \end{anm}
    \begin{alg}[Algorithmus zur JNF]
        \begin{enumerate}[Eingabe:]
            \item [\textbf{Eingabe:}] $A\in M_{n,n}(K)$, so dass $\chi_{A}^{\text{char}}$ in Linearfaktoren zerfällt.
            \item [\textbf{Ausgabe:}] JNF von $A$.
            \item [\textbf{Durchführung:}] 
            \begin{enumerate}[1.]
                \item Bestimme die nichtkonstanten Invariantenteiler $g_1,...,g_r$ von $A$.
                \item Bestimme die Primfaktorzerlegung 
                $$ g_i=(t-\lambda_{i,1})^{m_{i,1}}\cdot ... \cdot (t-\lambda_{i,k_i})^{m_i,k_i}, i=1,...,r$$
                \item Erhalte 
                $$A\approx \begin{pmatrix}
                    J(\lambda_{1,1},m_{1,1})& & \\
                    & \ddots & \\
                    & & J(\lambda_r,k_r,m_{r,k_r})
                \end{pmatrix}$$
                \item Gruppiere Jordanmatrizen zu gleichen EW zusammen (jeweils nach aufsteigender Größe geordnet)
            \end{enumerate}
        \end{enumerate}
    \end{alg}
    \begin{bsp}
        \begin{enumerate}[(a)]
            \item (vgl. Bsp 5.9 (b))
            $$ A=\begin{pmatrix}
                4 & -1 & -2 & 3 \\
                -1 & 5 & 2 & -4\\
                0 & 1 & 3 &-1\\
                1 & 2 & 2 & 1 
            \end{pmatrix}$$
            $\implies c_1(A)=1,c_2(A)=1,c_3(A)=t-3=:g_1,c_4(A)=(t-3)^2(t-2)=t^3-8t^2+21t-18=:g_2$
            Weierstrassteiler von $A$: 
            $$h_1=t-3,h_2=t-2,h_3=(t-3)^2$$
            \begin{align*}
                \implies A\approx B_{h_1,h_2,h_3}=\begin{pmatrix}
                    B_{h_1}& & \\
                    & B_{h_2}& \\
                    & &B_{h_3}
                \end{pmatrix} &\approx \begin{pmatrix} 
                    J(3,1)& & \\
                    & J_(2,1)& \\
                    & & J_(3,2)
                \end{pmatrix}\\
                &\approx \begin{pmatrix} 
                    J(3,1)& & \\
                    & J_(3,2)& \\
                    & & J_(2,1)
                \end{pmatrix}\\
                &= \begin{pmatrix}
                    3 & & & \\
                    & 3 & 0 &\\
                    & 1& 3 & \\
                    & & & 2
                \end{pmatrix}
            \end{align*}
            \item (vgl. 4.6)
            \begin{align*}
            A&=\begin{pmatrix}
                0 & 1 & 3\\
                3 & 1 & -4 \\
                -2 & 1 & 5
            \end{pmatrix}\in \M_{3,3}(\Q)\implies c_1(A)=1,c_2(A)=1,c_3(A)=(t-2)^3\\
            &\implies \text{Weierstrassteiler von }A: h_1=(t-2)^3\\
            &\implies A\approx B_{h_1}\approx J(2,3)=\begin{pmatrix}
                1 & 0 & 0\\
                1 & 2 & 0\\
                0 & 1 & 2 
            \end{pmatrix}\text{  } (\text{JNF von } A)
        \end{align*}
        \item \begin{align*}
        A&=\begin{pmatrix}
            1 & 1 & 2\\
            1 & 1 & -2 \\
            -1 & 1 & 4
        \end{pmatrix}\in\M_{3,3}(\Q)\implies c_1(A)=1,c_2(A)=t-2,c_3(A)=(t-2)^2\\
        &\implies \text{Weierstrassteiler von }A: h_1=t-2,h_2=(t-2)^2\\
        &\implies A\approx B_{h_1,h_2}=\begin{pmatrix}
            B_{h_1}&\\
            &B_{h_2}
        \end{pmatrix}\approx \begin{pmatrix}
            J(2,1)&\\
            & J(2,2)
        \end{pmatrix}=\begin{pmatrix}
            2 & & \\
            & 2 & 0\\
            & 1 & 2 
        \end{pmatrix}\text{  } (\text{JNF von }A)
    \end{align*}
        \end{enumerate}
        \end{bsp}
        
\end{document}